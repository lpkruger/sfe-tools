
\section{The $k$-means clustering algorithm}
\label{sec:k-means}

The $k$-means algorithm~\cite{pattern-classification,Llyod-82} is
shown in Figure~\ref{fig:k-means}. Assume that we are given $n$
samples $x_1,\cdots,x_n$, where each sample is a $m$-dimensional
vector of real numbers. The number of clusters is $c$. The algorithm
maintains $c$ means $\mu_1,\cdots,\mu_c$. Initially, assume that the
means are assigned arbitrary values. A sample $x_i$ is deemed to be in
the cluster $j$ if it is closest to the mean $\mu_j$, where mean of a
cluster $\{x'_1,\cdots,x'_r \}$ is
$\frac{x'_1+\cdots,x'_r}{r}$. Distance between two $m$-dimensional
vectors $x$ and $y$ is given by $\sum_{j=1}^m (x[j] - y[j])^2$, where
$x[j]$ is the $j$-th element of the vector $x$.  Other distance
metrics~\cite[Chapter 10]{pattern-classification}, such as scatter
metrics, can be used instead of the distance metric mentioned above.
Each iteration of the $k$-means algorithms recomputes the means and
reclassifies the samples. The algorithm terminates when it detects
``no change'' in the means. The precise definition of ``no change''
depends on the specific metric being used. We also assume that the
initial cluster means are chosen randomly. There is some research on
picking the initial cluster means~\cite{Initial:kmeans}. Various
techniques for picking initial cluster means can be easily
incorporated into our algorithm. This issue will not be discussed
further in the paper.


\begin{figure}
\begin{center}
\begin{programbox}
\mbox{Algorithm ($k$-means clustering)}
\BEGIN \mbox{initialize $n,c,\mathbf{\mu}_1,\cdots,\mathbf{\mu}_c$}
	\DO \mbox{classify $n$ samples according to nearest $\mathbf{\mu}_i$, and }
	    \mbox{recompute $\mathbf{\mu}_i$}
	\mbox{{\bf \underline{until}} no change in $\mathbf{\mu}_i$'s}
 \mbox{{\bf \underline{return}} $\mathbf{\mu}_1,\mathbf{\mu}_2,\cdots,\mathbf{\mu}_c$}
\END
\end{programbox}
\end{center}
\caption{The $k$-means clustering algorithm.}
\label{fig:k-means}
\end{figure}

\subsection{Distributed $k$-means}

Assume that Alice $A$ (party $1$) has $z$ samples $\{
x_1,\cdots,x_{n_A} \}$, and Bob $B$ (party $2$) has $n-n_A$ samples
$\{ x_{n_A + 1},\cdots,x_n \}$. Each party wants to jointly cluster
their samples without revealing any private information. We are
assuming that clustering the union of samples from the two parties is
more desirable than clustering the two samples individually.

Assume that there is a trusted third party $TTP$. $A$ and $B$ perform
iterations locally. However, at each iteration the new cluster means
$\mathbf{\mu}_i$s are computed by communicating with the $TTP$. Let
$C_i^A$ and $C_i^B$ be the cluster corresponding to mean
$\mathbf{\mu}_i$ for $A$ and $B$, respectively.  $A$ sends $c$-pairs
$\langle (a_1,b_1), \cdots, (a_c,b_c) \rangle$ to $TTP$, where $a_i =
\sum_{x_j \in C_i^A } x_j$ and $b_i = \mid C_i^A \mid$ ($a_i$ is the sum of
samples in cluster $C_i^A$ and $b_i$ is the number of samples in the
cluster $C_i^A$). Analogously, $B$ sends $c$-pairs $\langle (d_1,e_1),
\cdots, (d_c,e_c) \rangle$ to the $TTP$, where $d_i = \sum_{x_j \in
C_i^B } x_j$ and $e_i = \mid C_i^B \mid$. The $TTP$ computes the $c$
means $\langle \mu_1, \cdots, \mu_c \rangle$ and sends them to $A$ and
$B$, where $\mu_i = \frac{a_i+d_i}{b_i+e_i}$. We call this algorithm
{\em distributed $k$-means} or $D_{\mbox{$k$-means}}$.

\subsection{Assumptions}

Our goal is to design a privacy-preserving $k$-means that does not
use a TTP. Before we present such an algorithm, we state assumptions
made in the design of our  privacy-preserving algorithm.

\paragraph{Number of parties.} In this paper we only present the
two party case. 

\paragraph{The adversary model.} We assume a  semi-honest
adversary (also called honest but curious adversary model)~\cite{Goldreich:vol2}.
There are standard constructions that transform a protocol that
is secure in the semi-honest model and produce a protocol that is
secure in a more general malicious model (these constructions are
called ``semi-honest to malicious'' compilers, and details of these
constructions can be found in~\cite{Goldreich:compiler:99}).

\paragraph{Information disclosure.} Our privacy-preserving algorithm
discloses the cluster means at the various steps to the two parties.
Therefore, the computation of classifying samples according to the
nearest cluster means can be performed locally. Therefore, the
complexity of our privacy-preserving algorithm depends only on the
number of steps taken by the $k$-means algorithm and the
number of features, but not on the size of the data. This is a desirable property
because usually the data sets to be clustered can be very large.

\subsection{Privacy-preserving $k$-means}

In order, to create a privacy-preserving version of $k$-means that does
not use a TTP we have to devise a privacy-preserving protocol to
compute the cluster means. Consider the computation of a single
cluster mean $\mu_i$. Recall that in distributed $k$-means each
party sends $(a_i,b_i)$ and $(d_i,e_i)$ to the TTP, which computes
$\frac{a_i+d_i}{b_i+e_i}$; this is precisely the function for which
we have to devise a privacy-preserving protocol. This problem can be
formally defined as follows:

\begin{definition}
\rm
The {\em weighted average problem (WAP)} is defined as follows:
party $1$ has a pair $(x,n)$, where $x$ is a real number and $n$ is
a positive integer. Similarly, party $2$ has pair $(y,m)$. They want
to jointly compute $\frac{x+y}{n+m}$. In other words, we need a 
privacy-preserving protocol for the following functionality:
\begin{eqnarray*}
((x,n),(y,m)) & \longmapsto & (\frac{x+y}{n+m}, \frac{x+y}{n+m})
\end{eqnarray*}
The notation shown above means that the first and second party provide
inputs $(x,n)$ and $(y,m)$ to the protocol and both parties receive
output $\frac{x+y}{n+m}$. Notice that WAP is different than the classical
problem of computing the averages, where $n$ parties have a number and 
they jointly want to compute the average without revealing their individual
numbers. In the classical problem, the number of parties $n$ is known
to all the parties. In WAP, the number of points $n$ and $m$ needs to
be kept secret. 
\end{definition}

Let ${\cal P}_{WAP}$ be a privacy-preserving protocol for solving
WAP. Two protocols for WAP are presented in Section~\ref{sec:WAP}.  In
the privacy-preserving $k$-means algorithm (denoted as
$PP_{\mbox{$k$-means}}$) $A$ and $B$ use ${\cal P}_{WAP}$ instead of
the trusted third party $TTP$ to compute the cluster means
$\mathbf{\mu}_i$s.  The algorithm is shown in
Fig~\ref{fig:pp-k-means}. We only show the part of the algorithm
executing at Alice's (party $1$) side. Bob (party $2$) will execute a
similar algorithm at his side.

\noindent
{\bf Note:} Suppose that the initial clusters are picked randomly. For
the privacy-preserving algorithm we need a protocol for two parties to
jointly pick a common random vector. Such a protocol is called {\it
coin-tossing into the well} and is based on commitment schemes
(see~\cite[Section 7.4.3.1]{Goldreich:vol2}).

\begin{figure}
\begin{center}
\begin{programbox}
\mbox{Algorithm $PP_{\mbox{$k$-means}}$ (privacy-preserving $k$-means clustering)}
\BEGIN \mbox{initialize $n_A ,c,\mathbf{\mu}_1,\cdots,\mathbf{\mu}_c$}
	\DO \mbox{classify $n_A$ samples according to nearest $\mathbf{\mu}_i$}
	    \FOR i := 1 \TO c \STEP 1 \DO
	     \mbox{Let $C_i^A$ be the $i$-th cluster}
	     \mbox{Compute $a_i = \sum_{x_j \in C_i^A } x_j$ and $b_i = \mid C_i^A \mid$}
	     \mbox{recompute $\mathbf{\mu}_i$ by invoking the protocol ${\cal P}_{WAP}$ }
	    \OD
	\mbox{{\bf \underline{until}} no change in $\mathbf{\mu}_i$}
 \mbox{{\bf \underline{return}} $\mathbf{\mu}_1,\mathbf{\mu}_2,\cdots,\mathbf{\mu}_c$}
\END
\end{programbox}
\end{center}
\caption{The privacy-preserving $k$-means clustering algorithm.}
\label{fig:pp-k-means}
\end{figure}

\subsection{Proof of Privacy}

In this section we provide a proof of privacy for the protocol shown
in Figure~\ref{fig:pp-k-means}.  The proof uses a semi-honest
adversary model. Notice that in the distributed $k$-means algorithm
$\mathcal{D}_{\mbox{$k$-means}}$ both parties only know their input
and output.  Definition of privacy is based on the intuition that
parties should learn nothing more from the messages used in
privacy-preserving protocol, i.e., the messages received by a party
during an execution of a privacy-preserving protocol can be
``effectively computed'' by only knowing its input and output. This
idea is formalized below:

\begin{definition}
\label{def:privacy}
\rm
Let $x$ and $y$ be inputs of the two parties and $\langle f_1 (x,y),
f_2 (x,y) \rangle$ be the desired functionality, i.e., the first party
wants to compute $f_1 (x,y)$ and the second wants to compute $f_2
(x,y)$. Let $\Pi$ be a two-party protocol to compute $f$. The view
of the first party after having participated in protocol $\Pi$ 
(denoted by $\mbox{VIEW}_1^\Pi (x,y)$) is $(x,r,m_1, \cdots m_t)$, where
$r$ are the random bits generated by party $1$ and $m_1, \cdots, m_t$ is the
sequence of messages received by party $1$, while participating in protocol
$\Pi$. The view $\mbox{VIEW}_2^\Pi (x,y)$ for the second party  is defined
in an analogous manner.

We say that $\Pi$ {\em privately computes} $f$ if there exists 
probabilistic polynomial-time algorithms (PPTA), denoted by $S_1$ and $S_2$
such that

\begin{eqnarray*}
\{ S_1 (x,f_1 (x,y)) \}_{x,y} & \equiv^s & \{ \mbox{VIEW}_1^\Pi (x,y) \}_{x,y} \\
\{ S_2 (x,f_2 (x,y)) \}_{x,y} & \equiv^s & \{ \mbox{VIEW}_2^\Pi (x,y) \}_{x,y} \\
\end{eqnarray*}

In the equation given above, $\equiv^s$ denotes {\em statistically
indistinguishable}.  Two probability ensembles $X = \{ X_w \}_{w \in
S}$ and $Y = \{ Y_w \}_{w \in S}$ indexed by $S$ are statistically
indistinguishable if for some negligible function $\mu : \aleph \mapsto
[0,1]$ and all $w \in S$,
\begin{eqnarray*}
\sum_{\alpha} \mid Pr ( X_w = \alpha ) - Pr ( Y_w = \alpha ) \mid & < & \mu ( \mid w \mid )
\end{eqnarray*}
A function $\mu : \aleph \mapsto [0,1]$ is called {\it negligible}
if for every positive polynomial $p$, and all sufficiently large $n$'s,
$\mu (n) < \frac{1}{p(n)}$. There is a weaker notion of indistinguishability
called {\em computationally indistinguishable}. We will use statistical
indistinguishability throughout the paper, but all the results hold even if
the weaker notion of indistinguishability is used. Detailed definitions of
these concepts can be found in~\cite{GoldreichBookVol1,Goldreich:vol2}. 

\end{definition}

The privacy-preserving $k$-means algorithm uses the privacy-preserving
protocol ${\cal P}_{WAP}$ for the WAP. Assume that the two parties
invoke the protocol ${\cal P}_{WAP}$ as an oracle, i.e., both parties
write their respective inputs (in this case $(x,n)$ and $(y,m)$) and
invoke the oracle which returns the result (in this case
$\frac{x+y}{n+m}$). Recall that in the distributed $k$-means
algorithms both parties learn the cluster means at various steps. If
we use oracle calls to compute the cluster means, then the two parties
also learn only the cluster means. So the views in the two cases are
{\it identical}.  Hence, the conditions of
definition~\ref{def:privacy} are trivially satisfied. However, there
are additional messages exchanged in the protocol ${\cal P}_{WAP}$
used to compute the cluster means. We need to ensure that nothing can
be learned from these messages. The privacy of protocol shown in
Figure~\ref{fig:pp-k-means} follows from the composition
theorem~\cite{CanettiComposition} stated below ($g$ is the algorithm
shown in Figure~\ref{fig:pp-k-means} and $f$ is the protocol ${\it
P}_{WAP}$ to solve WAP described in Section~\ref{sec:WAP}):

\begin{theorem}
\label{thm:composition}
{\rm (Composition Theorem for the semi-honest model):}
Suppose that $g$ is privately reducible to $f$ and that there exists a protocol
for privately computing $f$. Then there exists a protocol for privately
computing $g$.
\end{theorem}



