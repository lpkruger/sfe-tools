%% Importance of privacy
%% cite laws and popular articles

\section{Contribution}
The previous chapter introduced our work in privacy-preserving
genomics by presenting algorithms for dynamic programming such
as the Smith-Waterman sequence alignment.  In this chapter,
we present a privacy preserving algorithm for clustering, an
unsupervised learning algorithm used in genomics and many other
machine-learning domains.

%% Contributions of this paper

This chapter makes the following contributions:

\begin{itemize}
\item We present the design and analysis of privacy-preserving $k$-means
clustering algorithm for horizontally partitioned data (see
Section~\ref{sec:k-means}). The crucial step in our algorithm is
privacy-preserving of cluster means. We present two protocols for
privacy-preserving computation of cluster means. The first protocol is
based on oblivious polynomial evaluation and the second one on
homomorphic encryption. These protocols are described in detail in
Section~\ref{sec:WAP}.

\item We present an open-source implementation of our algorithm. 
We believe that modular design of our implementation will enable other
researchers to use our implementation.  We evaluated the two privacy-preserving
clustering algorithms on real data sets. Our first conclusion is that
privacy-preserving clustering is feasible. For example, for a large
data set ($5,687$ samples and $12$ features) from the speech
recognition domain our homomorphic-encryption-based algorithm took
approximately $66$ seconds. We also observed that both in bandwidth
efficiency and execution overhead algorithms based on homomorphic
encryption performed better than the one based on oblivious polynomial
evaluation. A detailed discussion of our evaluation is given in
Section~\ref{sec:eval}.

\end{itemize}



\section{Related work}

As discussed in the previous chapter,
emphasis has been placed on preserving the
privacy of user-data aggregations, e.g., databases of personal
information.  Access to these  collections is, however, enormously
useful.  It is from this balance between privacy and utility that the
area of {\it privacy preserving data-mining}
emerged~\cite{Agrawal-Srikant,Lindell-Pinkas}.


%% Clustering description

Unsupervised learning deals with designing classifiers from a set of
unlabeled samples. A common approach for unsupervised learning is to
first cluster or group unlabeled samples into sets of samples that are
``similar'' to each other. Once the clusters have been constructed, we
can design classifiers for each cluster using standard techniques
(such as decision-tree
learning~\cite{Mitchell:AI,Quinlan:86}). Moreover, clusters can also
be used to identify features that will be useful for
classification. There is significant research on privacy-preserving
algorithms for designing
classifiers~\cite{Agrawal-Srikant,Lindell-Pinkas}. This paper
addresses the problem of privacy-preserving algorithms for clustering.


%% Problem description

Assume that Alice $A$ and Bob $B$ have two unlabeled samples $D_A$ and
$D_B$. We assume that each sample in $D_A$ and $D_B$ has all the
attributes, or the data sets are horizontally partitioned between $A$
and $B$. Alice and Bob want to cluster the joint data set $D_A \cup
D_B$ without revealing the individual items of their data sets (of
course Alice only obtains the clusters corresponding to her data set
$D_A$). In this paper, we assume that clustering the joint data set
$D_A \cup D_B$ provides better results than individually clustering
$D_A$ and $D_B$.  Using a large data set from the networking domain we also demonstrate
that clustering the joint data set results in significantly different
clusters than individually clustering the data sets (see end of
section~\ref{sec:eval} for details). We present a privacy-preserving
version of the $k$-means algorithm where only the cluster means at the
various steps of the algorithm are revealed to Alice and Bob.


%% Applications of clustering to medical informatics
%% and intrusion detection

There are several applications of
clustering~\cite{pattern-classification}. Any application of
clustering where there are privacy concerns is a possible candidate
for our privacy-preserving clustering algorithm. For example, suppose
network traffic is collected at two ISPs, and the two ISPs want to
cluster the joint network traffic without revealing their individual
traffic data. Our algorithm can be used to obtain joint clusters while
respecting the privacy of the network traffic at the two ISPs. An
application of clustering to network intrusion detection is presented
by Marchette~\cite{Marchette99}.  Clustering has been used for
forensics~\cite{Pouget:Dacier} and root-cause analysis for
alarms~\cite{Julisch:TISSEC}. Clustering has also been used in
bioinformatics. For example, Dhillon {\it et al.}~\cite{Dhillon:Bio}
have used clustering to predict gene function. We believe that
privacy-preserving clustering can be used in bioinformatics where the
data sets are owned by separate organizations, who do not want to
reveal their individual data sets.


