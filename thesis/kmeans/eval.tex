\section{Implementation and Experiments}
\label{sec:eval}

This section looks at the feasibility of our solution by evaluating
the cost of the protocol on real data-sets.  The goal of this study is
to establish the cost of our privacy-preserving clustering algorithms
on real applications. We principally seek to understand the
performance and privacy tradeoffs inherent to the operation of the
protocols.

We evaluated three clustering algorithms. The {\it simple} scheme is
used throughout as a baseline for our experiments.  This protocol
implements the $k$-means clustering algorithm as described in
section~\ref{sec:k-means}.  This algorithm does not use any
privacy-preserving protocols.  This represents the nominal cost of
clustering, and will be present in any $k$-means clustering approach,
independent of if and how privacy is implemented.  Throughout this
section {\it features} refer to the dimension of the vectors being
clustered and each iteration of the $k$-means algorithm is referred to
as {\it round}.  Our first privacy-preserving protocol (referred to as
{\it OPE}) uses oblivious polynomial evaluation. This protocol is
described in detail in Section~\ref{subsec:OPE}.  For oblivious
polynomial evaluation we use the protocol presented by Naor and
Pinkas~\cite{NaorPinkas99}. The next privacy-preserving protocol
(referred to as {\it HE}) uses homomorphic encryption scheme of
Paillier~\cite{Paillier99}. This protocol is described in detail in
Section~\ref{sec:homomorphic-encryption}.  

\paragraph{Implementation.}
Our system consists of approximately $3000$ lines of Java code, split
up into a number of self-contained modules.  The $k$-means algorithm
module implements actual clustering computations as described in
Section 3.  During each iteration, this module calls the {\small\sf
protocol} module to compute the cluster means for each dimension of
the cluster.  The {\small\sf protocol} module sets up the framework of
communication, and calls the specific protocol handlers with a common
interface, depending on which protocol is selected. In the {\it
simple} handler, Alice sends $(x,n)$ to Bob, who computes the cluster
mean $\frac{x+y}{n+m}$ and sends it to Alice.  The OPE and HE
protocol handlers implement the protocols described in
Sections~\ref{subsec:OPE} and~\ref{subsec:homomorphic}.

The central results uncovered by this investigation include:

\newcommand{\itembase}{\setlength{\itemsep}{-3pt}}

\begin{enumerate}
\itembase

\item
Clustering using HE is two orders of magnitude more bandwidth
efficient than OPE, and executes in 4.5 to 5 times less time.  This is
largely due to bandwidth and computational costs associated with the
oblivious transfers used by OPE.

\item 
Our protocols perform clustering with perfect fidelity; that is, the clusters
resulting from our algorithms are identical to those reported by a
$k$-means algorithm with no privacy for reasonable parameter choices.

\item
Small, medium, and large data-sets can be clustered efficiently.

\item
Costs scale linearly with feature and rounds.  The number of samples
affects run-time only inasmuch as it increases the number of rounds
toward convergence.

\item
Protocol parameters affect bandwidth usage by constant factor.
Moreover, exponential increases in security or supported message space
result in linear increases in execution run-times.

\end{enumerate}

\noindent
We begin in the following section by exploring several real data-sets
representative of expected environments.

\subsection{Experimental Data}
\label{sec:exdata}

The validity of our experimental approach is partially dependent on
the realism of our test data.  For this reason, we have obtained a
collection of externally provided data-sets representing diverse
applications.  All experiments described in this section use the
{\it synthetic}, {\it river}, {\it robot}, and {\it speech} data-sets
detailed below.

We selected the elements of our {\it synthetic} data-set to enable
testing and measure startup costs.  This data set includes 4 points
uniformly distributed within a 6 dimensional space.  By design, the
data clusters quickly into 4 "natural" clusters within 2 rounds under
the $k$-means algorithm in all experiments.

Originally used in the Computation Intelligence and Learning
(COIL) competition, the {\it river} data-set describes 
measurements of river chemical concentrations and algae
densities~\cite{coil99}.  The river data was used to ascertain the
summer algae growth of river water in temperate climates.  The
clustered data is used to inform the relationship between the presence
and concentrations of various chemicals in public waterways and algae
growth.  The river contains 184 samples with 15 features per sample.

The {\it robot} data-set~\cite{pion98} contains continuous sensor
readings from the Pioneer-1 mobile robot used for testing computer
learning and conceptual development approaches.  Each of the 697
samples contains 36 features from sensor arrays of the Pioneer-1
mobile robot.  The samples were taken every 100ms and reflect the
movements and changing environment in which the robot was tested.  The
data has been clustered in prior use to recognize experiences with
common outcomes.

The {\it speech} data-set~\cite{jvow00} documents the measured voice
characteristics of spoken Japanese vowels.  Nine male speakers uttered
two Japanese vowels $/ae/$ repeatedly.  Sampled at 10kHz, the 640
utterances resulted in 12 features of 5,687 samples.  This large
data-set is used in the context of our experiments to evaluate the
degree to which the proposed protocols scale with the size of the
input data.  Similar data-sets are clustered frequently to help guide
speech recognition software~\cite{kts99}.

Each of the data-sets represents a singular corpus.  In contrast, our
protocols are targeted for applications of clustering with two
parties.  We model the two party case by randomly subdividing the
samples into equal sized subsets and assigning them to each party.  In
real environments the size of the sets may be vastly different.
Our approximation approach ensures that this kind of asymmetry will be
transparent to both parties both in execution and performance.  That
is, the performance of the algorithm is largely independent of the
number of samples.  However, as we shall see below, the number
of features has tremendous effect on the cost of clustering.

The last data set (called the {\it ping} data-set) was collected by
us. The purpose of collecting this data was two fold:
\begin{itemize}
\item Test our clustering algorithm on a large data set. 
\item Construct a data set that can be naturally partitioned to demonstrate that
jointly clustering two data sets can produce significantly different
results than individually clustering them. 
\end{itemize}
We setup two hosts
(referred to as $A$ and $B$) to measure ICMP ping round-trip
times. There were $4$ ping targets located around the world (one of
the ping targets was on the same subnet as host $B$). On each host and
for each ping target the pings were grouped in blocks of $200$. For
each block, a $3$-tuple consisting of the following three values was
generated: the average time to live (TTL), the average round-trip time
(RTT), and fraction of lost packets (\%drop). We collected data over a
period of $24$ hours and generated a data set consisting of $23872$
data points, which were evenly divided between host $A$ and $B$. We
ran our clustering algorithm on the joint data set, and data sets
corresponding to hosts $A$ and $B$. 



\subsection{Experimental Setup}
\label{sec:exsetup}

We use the architecture and code described earlier for the
experiments described throughout.  All experiments are executed on a
pair of 3Ghz machines with 2 gigabyte physical memory.  The
experimental application is running on the Sun Microsystems Java
Virtual Machine version 1.5~\cite{sun04} on the Tao Linux version 1.0
operating system~\cite{tao04}.  The protocols are executed on a
100Mbps unloaded LAN with a measured round-trip time of 0.2
milliseconds.

\begin{comment}
Unless otherwise stated, the reported measurements represent the
averages taken over 100 experiments.
\end{comment}

The experiments profile the additional cost of providing privacy in
clustering sensitive data.  To this end, we focus on three metrics of
cost and utility; {\it communication overhead}, {\it delay}, and {\it
precision}.  Communication overhead records the amount of additional
network bandwidth used by the privacy schemes over the simple schemes.
Delay measures the additional time required to complete the
clustering.

Precision is used to measure the degree to which the approximated
clustering diverge from those reported by a simple $k$-means
algorithm, and is calculated as follows.  Let $X = \{x_1, \dots,
x_n\}$ be the sample data set to be clustered.  $C_1 \subseteq 2^X$ is
the clustering of $X$ by the simple algorithm, and $C_2 \subseteq 2^X$
is the clustering returned by the OPE algorithm (the HE metric is
defined similarly in the obvious manner).  For each pair $(x_i,x_j)$
such that $1 \leq i < j \leq n$ an error occurs if

\vspace{-4pt}

\itembase
\begin{enumerate}
\itembase
\item 
$x_i$ and $x_j$ are in the same cluster in $C_1$, but in $C_2$
they are in different clusters.

\item
$x_i$ and $x_j$ in the same cluster in $C_2$, but in $C_1$
they are in different clusters.
\end{enumerate}

\vspace{-4pt}

\noindent
The total number of errors is denoted $E$.  The maximum number of
errors is $N = n(n-1)/2$.  The precision $P$ is given by $(N-E)/N$.

Both OPE and HE have unique parameters which dictate the performance
and security of each protocol.  The performance of HE is most
effected by the size of the primes used to select the homomorphic
encryption keys.  Small primes can be cryptanalyzed, and large ones
can unnecessarily increase bandwidth use and computational costs.
Like RSA, linear increases in the size of the primes should result in
exponential security improvements.

We use interval arithmetic to approximate real numbers (see
section~\ref{sec:approximating-reals}).  The size of the message space in
HE and the finite-field in OPE are chosen to achieve the desired
precision.  Two crucial parameters in the
oblivious polynomial evaluation protocol of Naor and Pinkas are $D$,
the degree of the masking polynomial and $M$, the total number of
points used (details of this algorithm can be found
in~\cite{NaorPinkas99}). The sender's masking polynomial $D$ has
degree $k .  d$, where $d$ is the degree of the polynomial $P$
being evaluated and $k$ is the security parameter.  Since in our
algorithm the polynomial being evaluated is always linear, the
security parameter is simply $D$.  Increasing $D$ strengthens the
sender's security.  Only $D+1$ points are needed to interpolate, but
the receiver sends $(D+1) . M$ pairs of values to the sender.  Out
of each set of $M$ pairs, one of them is related to $\alpha$ (the
point the polynomial is being evaluated on), and the other $M-1$
values are random.  The $1$-out-of-$M$ oblivious transfer protocol
(denoted as $OT_1^M$) is repeated $D+1$ times to learn the required
value.  So, increasing $M$ strengthens the receiver's security.
Unless otherwise specified, we selected $D=7$ and $M=6$.  For brevity,
we do not consider $D$ or $M$ further.

\subsection{Results}
\label{sec:exres}

\begin{table}
\begin{center}
\begin{tabular}{|l|c|c|c|c|c|c|c|}
\hline
& & \multicolumn{3}{|c|}{\bf Communications Overhead}
& \multicolumn{3}{|c|}{\bf Delay} \\
\cline{3-8}
\multicolumn{1}{|c|}{\bf Test} & {\bf Rounds} & & {\bf bytes} & {\bf percent}
& & {\bf milliseconds} & {\bf percent} \\
& & \raisebox{1.5ex}[0pt]{\bf bytes} & {\bf feature/rnd} & {\bf increase}
& \raisebox{1.5ex}[0pt]{\bf milliseconds} & {\bf feature/rnd} & {\bf increase} \\
\hline
\hline
\multicolumn{8}{|l|}{\bf\it Synthetic ({\it 4 samples, 6 features})} \\
\hline
Simple & 2 & 5959     & 0       & 0\%      & 168   & 0      & 0\% \\
\hline
OPE    & 2 & 1497823  & 124322 & 25035.48\%  & 10147 & 831.58 & 5939.88\%  \\
\hline
HE    & 2 & 13580 & 635.08 & 127.89\% & 2135  & 163.9166667 & 1170.83\%  \\
\hline
\hline  
\multicolumn{8}{|l|}{\bf\it River ({\it 184 samples, 15 features})} \\
\hline
Simple & 16 & 74574 & 0 & 0\% & 772 & 0 & 0\% \\
\hline
OPE & 16 & 29916457 & 124241.17 & 40116.47\% & 176133 & 730.67 & 22715.16\% \\
\hline
HE & 16 & 234422 & 566.03 & 314.35\% & 38721 & 158.12 & 4915.67\% \\
\hline
\hline
\multicolumn{8}{|l|}{\bf\it Robot ({\it 697 samples, 36 features})} \\
\hline
Simple & 8 & 94005 & 0 & 0\% & 1348 & 0 & 0\% \\
\hline
OPE & 8 & 36569040 & 126649.42 & 38801.16\% & 212776 & 734.125 & 15684.57\% \\
\hline
HE & 8 & 269698 & 610.04 & 186.90\% & 47662 & 160.8125 & 3435.76\% \\
\hline
\hline
\multicolumn{8}{|l|}{\bf\it Speech ({\it 5,687 samples, 12 features})} \\
\hline
Simple & 33 & 143479 & 0 & 0\% & 4198 & 0 & 0\% \\
\hline
OPE & 33 & 49359739 & 124183.48 & 34402.07\% & 294694 & 733.57 & 6919.87\% \\
\hline
HE & 33 & 384644 & 509.00 & 268.08\% & 66101 & 156.3207071 & 1474.58\% \\
\hline
\hline
\multicolumn{8}{|l|}{\bf\it Ping ({\it 28,392 samples, 3 features})} \\
\hline
Simple & 9 & 11644 & 0 & 0\% & 2765 & 0 & 0\% \\
\hline
OPE & 9 & 3429688 & 126594.2 & 29354.55\% & 23767 & 777.8519 &   759.566\% \\
\hline
HE & 9 & 30633 & 703.29 & 163.07\% & 9694 & 256.63 & 250.59\% \\
\hline
\end{tabular}
\end{center}
\caption{Experimental Results - resource and precision results from
experiments over the three data sets.  The feature/round statistics
show the costs of per feature clustering in a single round of the
k-means algorithm, e.g., a single execution of the privacy preserving
WAP protocol.}
\label{tbl:results}
\end{table}

Our first battery of tests broadly profile the performance of OPE and
HE.  Shown in Table~\ref{tbl:results}, the most striking
characteristic of these experiments is that they demonstrate that OPE
protocols consume two orders of magnitude more network resources than
the HE protocols.  These costs can be directly attributed to the
oblivious transfer algorithms whose primitive cryptographic operations
require the transfer of many polynomials between hosts.  The total
bandwidth costs scaled linearly for both OPE and HE.  That is, the
bandwidth costs per feature/round are relatively constant for the
given data sets, where we observed 0.03\% variance in scaled bandwidth
usage in OPE and 9.36\% in HE.  Note that the bandwidth is ultimately
of limited interest; the worst case experiment only consumes 47
megabytes of bandwidth over two and a half minutes.  Hence, our
protocols would have visible impact only the slowest or busiest
networks.

A chief feature illustrated by the timing measurements is that HE is
much more time and bandwidth efficient than OPE.  Surprisingly, HE is
4.5 to 5 times faster on all the data-sets for the selected
parameters.  The reasons for this is that the underlying oblivious
transfers incur large message exchanges between the two parties.
Hence, in all experiments the limiting factors are bandwidth and
computation.\footnote{Early implementations of our protocols were
limited by the latency caused by many individual round-trips in the
protocol.  We optimized these these by parallelizing exchanges, where
possible.  This vastly improved protocol performance, and as a direct
result, bandwidth and and computation have since emerged as the
limiting factors.}  The efficiency of HE with respect to OPE further
shows fixed costs (startup) are likewise dominated by the underlying
privacy preservation operations.  Further, like the bandwidth costs,
the execution of each algorithm scale linearly with the number of
features and rounds, where each feature round requires 730 and 160
milliseconds for OPE and HE to complete, respectively.

The cost of privacy-preservation in large data-set clustering is
noticeable.  For example, a large data-set containing 5687 samples and
12 features takes HE just 66 seconds to cluster, as opposed to the
4.19 seconds required by its simple $k$-means counterpart.  Hence for
this experiment, HE algorithm incurs slowdown of a factor of 15 and the
more expensive OPE a factor of 70.  These results are, for most
applications, clearly within the bounds of acceptable performance.
This is particularly encouraging in the face of past attempts; circuit
implementations of vastly simpler operations (averaging very small
collections of data points) took tens of seconds to
complete~\cite{mnps04}.

% precision

For the parameters we selected the precision of our privacy-preserving
algorithms (HE and OPE) was $100\%$.  The reasons for this are
two-fold.  The parameter choices for HE resulted in a message space
of $3^{40}$ values, which allowed us to map cluster means to 4 decimal
places.  Moreover, the data range was small in all our data-sets.
Hence, the error rounding caused by using interval arithmetic was
inconsequential.  Note that in other environments, where the message
space is required to be smaller (likely for performance reasons) or
the range of data values is large, precision errors may be introduced.

The costs of OPE grow slightly with increases in $D$ and $M$.  We
experimented with varied parameters of $D$ and $M$ equal 5, 10, 15 on
all the non-synthetic data-sets (for a total of 27 experiments) .  In
all cases increased cost was nominal; the parameter sets slowed the
performance of the algorithm down between 60\% and 190\% over a
baseline experiment, i.e., $M=D=5$.  Again, these costs are a direct
reflection of the costs of the underlying oblivious transfer.  Not
shown, the bandwidth costs in HE scale by a constant factor
proportional to $D$ and $M$.

\begin{figure}
\begin{center}
\fbox{\rotatebox{0}{\epsfysize=4.0in \epsfbox{kmeans/figures/dpe-param.pdf}}}
\caption{HE runtime costs by message space - in {\it
milliseconds}, the time to cluster the sample data-sets with various
widths of $n$ message spaces.}
\label{fig:dpe-param}
\end{center}
\end{figure}

As illustrated in Figure~\ref{fig:dpe-param}, increases the size $n$
(which is a product of two primes) in HE has modest affect on the
performance of the protocols.  Exponential increases in $n$ result in
linear increases in message size.  Because the network is a limiting
factor, such increases are, as shown, reflected in linear slowdowns.
Hence, very large intervals or high precision clustering can be
supported by small increases in bandwidth consumption.  As in OPE,
bandwidth costs in HE scale by a constant factor in these
experiments, where each protocol exchange increases directly in
proportion to the size of the primes.

For the ping data set our clustering algorithm generated $4$ clusters,
which correspond to the four target hosts. The centers for the four
clusters are shown in Figure~\ref{fig:kmeans-clusters}. As can be clearly
seen from the results, clusters found by the algorithm using the joint
data set are significantly different than the clusters found in the
individual data sets. Therefore, if the goal is to estimate RTT, TTL,
and \%drop for the target hosts to be used in networking applications
(such as routing), then clustering on the joint data set is desirable.

\begin{figure}
\begin{center}
\begin{tabular}{|l|c|} \hline
  & Cluster centers \\ \hline
$A$ &  {\small $(241.76,32.69,0.18)$, $(48.00,75.87,0.58)$, $(243.00,59.81,0.15)$, $(64.00,0.19,0.00)$}\\ \hline
$B$ &  {\small $(47.00,88.60,0.74)$, $(251.92,4.73,0.19)$, $(242.00,48.01,2.70)$, $(133.67,485.77,13.78)$}\\ \hline
Joint & {\small $(245.26,28.73,0.60)$, $(47.51,82.13,0.66)$, $(133.67,485.77,13.78)$, $(64.00,0.186,0.00)$}  \\ \hline
\end{tabular}
\end{center}
\caption{(TTL,RTT,\%drop) centers for the four clusters.}
\label{fig:kmeans-clusters}
\end{figure}
