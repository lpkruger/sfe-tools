\section{Privacy-Preserving Protocol for \\ the Weighted Average Problem}
\label{sec:WAP}

In the weighted average problem (WAP) we want to find
a privacy-preserving protocol for the following functionality:
\begin{eqnarray*}
((x,n),(y,m)) & \longmapsto & (\frac{x+y}{n+m}, \frac{x+y}{n+m})
\end{eqnarray*}
Recall that a protocol for WAP was used in
the privacy-preserving $k$-means algorithm (see Figure~\ref{fig:pp-k-means}).


A simple strategy to address this problem is to first approximate the
function $\frac{x+y}{n+m}$ by a circuit $C$, and then use standard
constructions~\cite{GMW87,Goldreich:JACM:91,Yao86} to construct
a privacy-preserving protocol.  Protocols constructed using this
strategy have a very high computational overhead. Malkhi {\it et al.} 
considered the cost of implementing these protocols in their work in
the Fairplay system~\cite{mnps04}.  They found that the protocol was
feasible for small circuits, e.g., a single $\wedge$-gate could be
implemented in $410$ milliseconds, and more complex integer numerical
functions could be implemented on the order of seconds.  They further
showed the runtimes of these protocols grow quickly with the size of
the input and complexity of the implemented function.  The most
complex function discussed by the authors computed a median of two
ten-element integer input sets.  This function took over $7$ seconds
to execute in a LAN environment, and over $16$ seconds in an WAN
environment.  The circuit for computing $\frac{x+y}{n+m}$ is
significantly more complex. Hence, with a non-trivial data set, a
single computation of cluster means may take several minutes to compute.  Note that
the underlying costs of Fairplay are not artifacts of the design, but
simply the cost of implementing the standard protocols; the reported
costs were almost completely dominated with circuit setup and the
necessary oblivious transfers.

In this section, we present two privacy-preserving protocols for WAP
that are more efficient than the standard protocols. The first
protocol is based on oblivious polynomial evaluation and the second on
homomorphic encryption. 
%Similarity of WAP with a problem that occurs
%in protocols for generation of shared RSA
%keys~\cite{Boneh-Franklin-2001,Gilboa99} is discussed in
%appendix~\ref{sec:shared-RSA}.


\subsection{Protocol based on oblivious polynomial evaluation}
\label{subsec:OPE}

We will first give a privacy-preserving protocol for a general problem,
and then at the end of the subsection demonstrate how we can construct
a privacy-preserving protocol for WAP.  Consider the following
problem.

\begin{definition}
\rm
Let ${\cal F}$ be a finite field. Party $1$ has two polynomials $P$
and $Q$ with coefficients in ${\cal F}$. Party $2$ has two points
$\alpha$ and $\beta$ in ${\cal F}$. Both parties want to compute
$\frac{P(\alpha)}{Q(\beta)}$. In other words, we want to privately
compute the following functionality:
\begin{eqnarray*}
((P,Q),(\alpha,\beta)) & \longmapsto & (\frac{P(\alpha)}{Q(\beta)},\frac{P(\alpha)}{Q(\beta)})
\end{eqnarray*}
We call this problem {\em private rational polynomial evaluation (PRPE)}.
\end{definition}

The protocol $\mathcal{P}_{PRPE}$
uses a protocol for oblivious polynomial evaluation, which is defined below.

\begin{definition}
\rm
Let $\mathcal{F}$ be a finite field.  The {\em oblivious polynomial
evaluation} or {\em OPE} problem can be defined as follows: Alice
$A$ has a polynomial $P$ over the finite field $\mathcal{F}$, and
Bob $B$ has an element $x \in \mathcal{F}$. After executing the  protocol
implementing OPE $B$ should {\em only know} $P(x)$ and $A$ should
know nothing.
\end{definition}

A protocol to solve the OPE was  given by Naor and
Pinkas~\cite{NaorPinkas99}.  Let $\mathcal{P}_{OPE}(P,\alpha)$ denote
the privacy-preserving protocol for OPE. We provide a protocol
$\mathcal{P}_{PRPE}((P,Q),(\alpha,\beta))$ for PRPE, which uses
$\mathcal{P}_{OPE}(P,\alpha)$ as an oracle. The protocol is shown in
Figure~\ref{fig:PRPE}. 

\begin{figure}
\framebox{\parbox[c]{6.5in}{
{\bf (Step 1)} Party $1$ picks a random element $z \in \mathcal{F}$ and
computes two new polynomials $zP$ and $zQ$. In other words,
party $1$ ``blinds'' the polynomials $P$ and $Q$.
\medskip\\
{\bf (Step 2)} Party $2$ computes $z P(\alpha)$ and $z Q(\alpha)$ by
invoking the protocol for OPE twice, i.e., invokes the 
protocol $\mathcal{P}_{OPE}(zP,\alpha)$ and $\mathcal{P}_{OPE}(zQ,\beta)$.
\medskip\\
{\bf (Step 3)} Party $2$ computes $\frac{P(\alpha)}{Q(\beta)}$ by computing
$\frac{z P(\alpha)}{z Q(\beta)}$ and sends it to party $1$.
}}
\caption{Protocol for PRPE.}
\label{fig:PRPE}
\end{figure}


\begin{theorem}
\rm
\label{thm:privacy-PRPE}
Protocol $\mathcal{P}_{PRPE}((P,Q)(\alpha,\beta)$ shown in Figure~\ref{fig:PRPE}
is privacy-preserving protocol for PRPE.
\end{theorem}
{\bf Proof:} The views of the two parties are
\begin{eqnarray*}
\mbox{VIEW}_1^{\mathcal{P}_{PRPE}} (P,Q) & = & (P,Q,\frac{P(\alpha)}{Q(\beta)}) \\
\mbox{VIEW}_2^{\mathcal{P}_{PRPE}} (\alpha,\beta) & = & (\alpha,\beta, z P(\alpha), z Q(\beta))
\end{eqnarray*}
The view of party $1$ consists of its input $(P,Q)$ and output $\frac{P(\alpha)}{Q(\beta)}$.
Therefore, there is nothing to prove (see definition~\ref{def:privacy}, we can use
$S_1$ as the identity function). The input and
output of party $2$ are $(\alpha,\beta)$ and $\frac{P(\alpha)}{Q(\beta)}$ respectively.
We have to show a PPTA $S_2$ such that
$S_2 (\alpha,\beta,\frac{P(\alpha)}{Q(\beta)})$ and $\mbox{VIEW}_2^{\mathcal{P}_{PRPE}} (\alpha,\beta)$
are statistically indistinguishable. Let $z'$ be a random element of $\mathcal{F}$ and $S_2 (\alpha,\beta,\frac{P(\alpha)}{Q(\beta)})$
be defined as follows:
\[
(\alpha,\beta, z' \frac{P(\alpha)}{Q(\beta)},z')
\]
It is easy to see that the following two ensembles are statistically indistinguishable:
\[
\begin{array}{l}
(\alpha,\beta, z' \frac{P(\alpha)}{Q(\beta)},z') \\
(\alpha,\beta, z P(\alpha), z Q(\beta))
\end{array}
\]
The reason is that if $z$ is a random element of $\mathcal{F}$ then $z
Q(\beta)$ is a random element of $\mathcal{F}$ as well. Moreover, the
ratio of the third and fourth elements in the view of party $2$ is
$\frac{P(\alpha)}{Q (\beta)}$, i.e., the output and the third element
of the view determine the fourth element of the view.

Recall that $\mathcal{P}_{PRPE}$ uses the protocol $\mathcal{P}_{OPE}$.
Using the composition theorem we conclude that $\mathcal{P}_{PRPE}$ 
is privacy preserving.
$\Box$

\paragraph{Protocol for WAP.} First, we show that a protocol $\mathcal{P}_{PRPE}$ for PRPE can be
used to solve WAP. Recall that in WAP party $1$ and party $2$ have inputs
$(x,n)$ and $(y,m)$ respectively. In the invocation of $\mathcal{P}_{PRPE}$, party $1$
constructs two polynomials $P(w) = w + x$ and $Q(w) = w + n$, and party $2$
sets $\alpha = y$ and $\beta = m$. The output both parties receive is equal
to $\frac{x+y}{n+m}$, which is the desired output. The proof of privacy
for this protocol follows from Theorem~\ref{thm:privacy-PRPE} and the composition
theorem.


\subsection{Protocol based on homomorphic encryption}
\label{subsec:homomorphic}

Homomorphic encryption and the Paillier cipher
was introduced in section \ref{sec:homomorphic-encryption}.
Party $1$ and $2$
have a pair of messages $(x,n)$ and $(y,m)$. The two parties want to
jointly compute $\frac{x+y}{n+m}$ in a privacy-preserving way. Assume
that party $1$ sets up a homomorphic encryption scheme $(G,E,D,M)$, and
publishes the public parameters $G$. 
The protocol $\mathcal{P}_H$ for WAP is shown
in Figure~\ref{fig:protocol-homomorphic}.

\begin{figure}
\framebox{\parbox[c]{6.5in}{
\begin{itemize}
\item {\bf (Step 1)} Party $1$ encrypts $x$ and $n$ and sends the encrypted values $x_1 \in E(x)$
and $n_1 \in E(n)$ to party $2$.

\item {\bf (Step 2)} Party $2$ computes a random message $z \in M$, and encrypts $z \cdot y$ and $z \cdot m$ to obtain $z_1 \in E(z \cdot y)$
and $z_2 \in E(z \cdot m)$.
Party $2$ computes the following two messages and sends it to party $1$:
\begin{eqnarray*}
m_1 & = & f(x_1^z,  z_1) \\
m_2 & = & f(n_1^z, z_2) \\
\end{eqnarray*}

\item {\bf (Step 3)} Using the two properties of the homomorphic encryption scheme $(G,E,D)$,
we have the following:
\begin{eqnarray*}
m_1 & = & E(z \cdot x + z \cdot y) \\
m_2 & = & E(z \cdot n + z \cdot m) \\
\end{eqnarray*}
Therefore, party $1$ can compute $ z (x+y)$ and $z (n+m)$, and hence can compute
$\frac{x+y}{n+m}$. Party $1$ sends $\frac{x+y}{n+m}$ to party $2$. 
\end{itemize}
}}
\caption{Protocol for WAP based on homomorphic encryption.}
\label{fig:protocol-homomorphic}
\end{figure}

\begin{theorem}
\label{thm:privacy-homomorphic}
\rm
Assume that the homomorphic encryption scheme $(G,E,D)$ is semantically
secure.
$\mathcal{P}_H((x,n),(y,m))$ is a privacy-preserving
protocol to compute$\frac{x+y}{n+m}$.
\end{theorem}
\input{kmeans/proof1}

The complexity of encryption and decryption operations of a scheme
$(G,E,D,M)$ depends on size of the message space $M$. Therefore, in
order to keep the complexity low it is important that the size of the
message space be small. However, in order to achieve adequate
precision the message space should be large. Chinese remainder theorem
(CRT) allows us to perform computation over smaller spaces and then
reconstruct the result for a larger message space. Let
$p_1,\cdots,p_m$ be $m$ small primes.  The two parties execute the
protocol described above for $Z_{p_1}, \cdots, Z_{p_m}$. Party $1$
receives $z (x+y)$ and $z (n+m)$ modulo $p_i$ (for $1 \leq i \leq
m)$. CRT allows party $1$ to reconstruct $z (x+y)$ and $z (n+m)$
modulo $N \; = \; \prod_{i=1}^m p_i$. This technique is also used by
Gilboa~\cite{Gilboa99}.

