\begin{figure*}[t]
  \centering

  \begin{tabular}{|rcl|}
    \hline
    \underline{Client} & & \underline{Server} \\
    $(U)$: username &
    $\xrightarrow{\mathrm{\texttt{SSH\_MSG\_USERAUTH\_REQUEST}}(\texttt{"sfeauth"},U)}$ & \\
    & $\xleftarrow{\mathrm{~~~~~~OT~choices~~~~~~}}$ & $H(\mathrm{password})$ \\
    & $\xrightarrow{\mathrm{~~~~~~~OT~keys~~~~~~~~}}$ & \\
    & $\xrightarrow{\mathrm{~~~~garbled~circuits~~~~}}$ & \\
    & $\xleftarrow{\mathrm{~~~~~chosen~circuits~~~~~}}$ & \\
    $\mathrm{password}$ & $\xrightarrow{\mathrm{~~~~~~~~~~keys~~~~~~~~~~}}$ & verify and evaluate \\
    & & \\
    & & \textbf{Success?} \\
    \textit{Success} & $\xleftarrow{\mathrm{\texttt{SSH\_MSG\_USERAUTH\_SUCCESS}}}$ & If yes \\
    \textit{Terminate} & $\xleftarrow{\mathrm{\texttt{SSH\_MSG\_USERAUTH\_FAILURE}}}$ &
    If no \\
    \hline
  \end{tabular}

  \caption{The SFE user authentication protocol 2.}
  \label{fig:sfeauth}
  \vspace{-1.5em}
\end{figure*}

\section{Implementation and Experiments}
\label{sec:eval}

We modified an existing open-source SSH client and server to use each
of the protocols described in Section~\ref{sec:proto}, and took
several performance measurements to evaluate the feasibility of our
approach. Our findings can be summarized as follows:
\begin{itemize}
\item Protocols that are secure in the semi-honest model, which is only 
secure against passive adversaries, can be executed very quickly.
\item Making the protocols secure against active adversaries
increases authentication time substantially, depending on the size of
the authentication circuit and the security parameter.  For example,
calculating an MD5 hash using 90 circuits increases the authentication
time from around 2 seconds to 12 seconds.  Although this may seem
tedious for some users, we achieve a high level of security with only
modest delays on inexpensive modern hardware.  We conclude that the technique
can achieve a favorable balance between efficient practicality and high
security.

\item Due the simplicity of private equality testing, Protocol 1 (the Straw man protocol)
can run extremely quickly even under the covert model at the expense
of resisting impersonation by an adversary who has gained knowledge of
a user's password hash.  If this security requirement can be relaxed
(for example, in an environment where passwords are stored on the
server using an identity hash, i.e. in plaintext) protocol 1 could be
a useful as high-speed authentication protocol.
\end{itemize}

\subsection{Implementation}

We implemented the protocols by modifying the Dropbear 0.52 SSH client
and server to support a new authentication protocol, to which we
assigned the name {}``sfeauth'' in the SSH authentication protocol
namespace. The scheme used for incorporating the protocols into the
SSH protocol is shown in Figure \ref{fig:sfeauth}. The Yao
sub-protocol is conducted through the encrypted SSH tunnel using a
reserved message which we dubbed
\texttt{SSH\_MSG\_USERAUTH\_SFEMSG}. If at any time the server detects
a cheating attempt by the client, the server fails the authentication
and terminates the protocol.

The MD5 and private equality circuits were implemented using a
prototype circuit compiler we developed first described in
\cite{1398071}, which also contains an embeddable implementation of
the Yao {}``garbled circuit'' protocol. The protocol was extended
using the techniques due to Lindell and Pinkas
\cite{lindellpinkas-eurocrypt07} to add resistance to malicious
parties in the covert model. The implementation also uses the
oblivious transfer protocol due to Naor and Pinkas
\cite{Naor-Pinkas:2001}.  All of the Yao protocol and authentication
code was written in C++ and integrated with the Dropbear SSH
client and server.

\subsubsection{Optimizations}

To improve performance of the protocols, we introduced several
optimizations to our implementation.
\begin{enumerate}
\item The client computes the garbled circuits used in the
authentication protocol in advance of the online protocol.  By
precomputing and storing garbled circuits, the time spent in this CPU
intensive step is removed from the user's perceived wait time to login
to a server.
\item We implemented an optimization described by 
Goyal \textit{et al.}~\cite{goyalmohasselsmith-eurocrypt08}.  In
constructing the garbled Yao circuits, the client generates a set of
seeds for a cryptographically-secure pseudorandom number generator.
The circuits are garbled using this PRNG, with one seed per circuit,
and hashes of the circuits are sent in place of the whole circuits.
After the server has chosen which circuits to evaluate, the seeds for
the non-chosen circuits are revealed to the server, who then uses the
PRNG to reconstruct the garbled circuit and verify the hash values,
and only the circuits to be evaluated are transferred in full.  This
saves many megabytes of wire communication, improving the overall
protocol performance.
\end{enumerate}

The prototype SSH client and server, as well as further documentation,
can be downloaded from our project
website.\footnote{\url{http://www.cs.wisc.edu/~lpkruger/ssh}}


\subsection{Experiments}

We conducted several usage experiments to measure the performance of
the authentication, and determine its feasibility in real
settings. Note that the semi-honest version is not secure for
real-world usage where the possibility of active malicious adversaries
cannot be ruled out, but the experiment is useful to establish an
upper bound for the potential performance with further
optimization. The tests were performed over a local network using
computers with eight core Intel Xeon processors and 8GB of RAM.

The performance results of our experiments are shown in
Table~\ref{table:performance}. The first row corresponds to the
semi-honest version of the protocol, and is the time on which the
\emph{ratio to semi-honest} column for other rows is based. The column
titled \emph{probability of attack success} refers to the probability
of a malicious client successfully convincing the server that he has
the proper credentials to authenticate. This calculation is discussed
in detail in section~\ref{sect:mainproto}. As our results indicate,
the time required to complete the protocol increases linearly as the
number of circuits increases, while the security guarantee increases
exponentially in this measure. Note that for less than an order of
magnitude increase over semi-honest implementation, sufficient
security guarantees for many practical settings can be attained. These
trends indicate that our technique is suitable for common use in real
applications.

\begin{table}[t]
\centering
\begin{tabular}{|l|c|c|c|}
\hline
{\it \# Circuits} & {\it Online Time (seconds)} & {\it Ratio to
Semi-Honest} & {\it Probability of attack success} \\
\hline\hline
1 ({\it Semi-Honest}) & 0.52 & 1.0 & 100\% \\
\hline
30 & 3.99 & 7.7 & $1.86 \times 10^{-7}$\% \\
\hline
60 & 7.86 & 15.2 & $1.73 \times 10^{-16}$\% \\
\hline
90 & 12.35 & 23.8 & $1.62 \times 10^{-25}$\% \\
\hline
120 & 16.46 & 31.8 & $1.50 \times 10^{-34}$\% \\
\hline
150 & 20.57 & 39.6 & $1.40 \times 10^{-43}$\% \\
\hline
\end{tabular}
\vspace{1em}
\caption{Wall-clock performance and security guarantees for the
optimized protocol in both semi-honest and covert settings. All times
are given in seconds.}
\label{table:performance}
\end{table}

We note that the performance is sensitive to available processing
power due to the many cryptographic primitives employed.  For example,
our implementation takes advantage of parallelization on multi-core
processors to encrypt multiple circuits in parallel as the server
performs its circuit verifications.  Due to the independence
of the verification of each circuit, parallel scaling can be achieved that
is extremely efficient with respect to available processors, potentially
allowing high security authentication with minimal delays to clients on 
server machines with enough processing power.

Overall, we believe that the results we have achieved so far
demonstrate the potential of this technique as a practical and secure
addition to the body of research in secure password authentication.
