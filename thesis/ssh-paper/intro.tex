\section{Introduction}
\label{sec:intro}

%% SSH is a widely-used protocol. As such, it is often used in settings
%% where sophisticated authentication mechanisms are not available.

Originally designed as a secure alternative to telnet, SSH has since
evolved into a layered protocol that serves as the secure transport
layer over which many other protocols execute. This functionality has
simplified the task of providing sophisticated cryptographic security
to a number of applications that need it. Unfortunately, it has also
encouraged SSH deployment in settings where strong authentication
mechanisms are not available, or worse yet, take a backseat to more
convenient measures such as interactive password login. This poses a
significant problem, as many casual SSH users may assume that the mere
presence of SSH guarantees security, unaware of the risks associated
with password authentication and the improper use of public key
cryptography~\cite{DBLP:journals/compsec/YangS99}. For example, in
Section~\ref{sec:overview} we describe a man-in-the-middle attack on
SSH password authentication.

\vspace{1ex}
\noindent
To address the attack described in Section~\ref{subsec:ssh-mitm}.  we
design a practical, yet cryptographically secure protocol for
password-based authentication and key establishment in SSH.  Even though
we use our protocol in the context of SSH, our technique can be applied 
to any scenario where password-based authentication is applicable. An
implementation of our protocol is available at \url{http://www.cs.wisc.edu/~lpkruger/ssh}. Our protocol
satisfies three important design principles.

\begin{description}

\item[(1) Compatible with legacy infrastructure.] Our protocol is
compatible with existing password authentication infrastructures.  It
does not require any changes to \emph{legacy servers} beyond upgrading
the SSH software and is thus deployable in common settings.  The use
of cryptographic hash databases to store passwords is common practice
on both Unix and Windows systems~\cite{smith08}. Typical Linux systems
(current versions of Ubuntu~\cite{ubuntu-shadow},
RedHat~\cite{redhat-shadow}, and Debian~\cite{debian-shadow})
typically use either the MD5 or SHA-512 hash function, with salts and
iterated rounds for added security against offline brute-force
attacks. Current Windows versions use a proprietary technology known
as the NT Hash~\cite{nt-hash}, but the principle is identical. Our
protocol is specifically designed to support the use of hash functions
to store passwords.

By contrast, other solutions for password-authenticated key exchange
require users to re-generate passwords, which greatly limits their
deployability.  They cannot be installed on legacy servers with large
existing user bases.  Some also require additional information to be
stored on the server or assume the existence of public-key
infrastructure (PKI).

\item[(2) Does not decrease security of password storage.] At the very
least, the password authentication mechanism should not provide weaker
security guarantees than the current system, in which users' passwords
are stored on the server in hashed form.  If the server stores
passwords in the clear, a compromise of the server will reveal the
passwords of all users.  Even without an external compromise, a
malicious server operator may impersonate a user in other
authentication domains.

Our protocol takes as inputs the password from the user and the hashed
password from the client (it is essential that the user's input into
the protocol is the actual password and not a hash; otherwise, a
malicious server operator could impersonate the user).  Therefore,
from the viewpoint of password security, it is as strong as existing
solutions, while providing significantly more protection against
man-in-the-middle attacks.

\item[(3) Enables derivation of a secure, shared cryptographic key.]
Our protocol enables the user and the server to derive a shared
cryptographic key(s) which can be used to protect their subsequent
communications.  The key remains secure (\ie, indistinguishable from
random) even in the presence of a malicious man-in-the-middle
adversary.  Unlike existing methods for password authentication in
SSH, our protocol does not require the user to check the validity of
the server's public key by manually verifying its fingerprint (we
argue that this requirement is largely ignored in practical deployment
scenarios).

Against an active adversary, the protocol is as secure as can be hoped
for in the case of password-based authentication.  It does not leak
any information except the outcome of an authentication attempt, \ie,
for any given password, the adversary can check whether the password
is correct.  Brute-force password-cracking remains feasible, but every
attempt requires executing an instance of the protocol.

\end{description}



\vspace{1ex}
\noindent
\textbf{Exploiting the special features of password authentication.}  
Our protocol uses Yao's ``garbled circuits'' protocol for secure function
evaluation (SFE) as a basic building block.  SFE is used to compute the
hash of the SSH client's password and compare it for equality with the
hash value provided by the SSH server.

Yao's original protocol is only secure against passive or semi-honest
adversaries~\cite{lindellpinkas-jcs,yao-focs86}, \ie, under the assumption
that all participants faithfully follow the protocol.  This model is
clearly unsuitable for SSH, which must be secure even if one of the
participants maliciously deviates from the protocol specification.
This includes the case when a malicious SSH client---who constructs
the garbled circuits in our protocol---deliberately creates a faulty
circuit in an attempt to learn the server's input into the protocol.
For example, the client may put malformed ciphertexts into the rows of
the garbled truth table which will only be evaluated when a certain
input bit from the server is equal to ``1,'' and correct ciphertexts
into the rows which will be evaluated when this bit is equal to ``0.''
By observing whether the server's evaluation of this circuit fails or
not, the malicious client can learn the value of the bit in question.
The malicious client may also submit a circuit which computes something
other than the hash-and-check-for-equality function required by SSH
authentication.

Yao's protocol can be modified to achieve security
against malicious parti\-cipants---either via cut-and-choose
techniques~\cite{lindellpinkas-eurocrypt07,woodruff-eurocrypt07}, or via
special-purpose zero-knowledge proofs~\cite{jareckishmatikov-eurocrypt07}
which enable the server to verify that the circuit is well-formed---but
the resulting constructions, while more efficient than generic
transformations, are still too expensive for practical use.

Our SFE-based construction in this paper exploits the special structure
of the authentication problem in a fundamental way.  The purpose of
the password authentication subprotocol in SSH is to compute a single
bit for the client: whether the hash of the password submitted by the
client is equal to the value submitted by the server or not.  The standard
cut-and-choose construction for SFE in the malicious model requires that
the server evaluate several garbled circuits submitted by the client and
the majority of them must be correct~\cite{lindellpinkas-eurocrypt07}.
In the context of password authentication for SSH, it is sufficient that a
\emph{single} circuit is correct.  Even if all but one circuits evaluated
by the server are faulty, a malicious client does not learn any more
than he would have been learned simply by submitting a wrong password.

% Our key observation in this paper is that full security in the
% malicious model is not necessary for the SFE subprotocol when used for
% password-based authentication inside SSH.  

Our key observation is that to prevent a malicious client from
authenticating without the correct password, it is sufficient for the
SSH server to either (a) detect that one of the circuits submitted by
the client is incorrect, or (b) evaluate at least one correct circuit.
In other words, the SSH server either detects the client's misbehavior or
rejects the client's candidate password because its hash does not match
the server's value.  In either case, authentication attempt is rejected.

We prove the security of our protocol against malicious
clients in a (modified) \emph{covert} model of secure
computation~\cite{aumannlindell,goyalmohasselsmith-eurocrypt08}.
Security in the covert model guarantees that any deviation from the
protocol will be detected with a high probability.  In our proof, instead,
we show that, with high probability, either the deviation is detected,
or the protocol computes the same value as it would have computed had the
client behaved correctly.  Security in this model can be achieved at a
lower cost than ``standard'' security against malicious participants,
enabling significant performance gains for our implementation viz.\
off-the-shelf SFE.

Security of an honest client against a malicious SSH server follows
directly from the security of the underlying oblivious transfer (OT)
protocol against malicious choosers, since the server's input into
the protocol is limited to his acting as a chooser in the OT executed
as part of Yao's protocol.  While the server can always perform a
denial-of-service attack by refusing to communicate the result of
authentication to the client, this is inevitable in any client-server
architecture.

The protocol is also secure against replay attacks since a
man-in-the-middle eavesdropper on an instance of the protocol does not
learn anything about the client's input (password), server's input
(password hash), or the shared key established by the client and
the server.

%% When password authentication is used, steps must be taken to prevent a
%% MITM attack wherein a malicious third party learns the client's
%% password. SSH uses host keys, which introduce a new set of problems
%% that are not trivial.


\vspace{1ex}
\noindent
\textbf{SPAKA protocols.}  Bellovin and Merritt pioneered a class of
protocols that use the client password as a shared secret for mutual
authentication~\cite{bellovin92}. These protocols, commonly referred
to as SPAKA (\emph{Secure Password and Key Authentication}) or PAKE
(\emph{Password Authenticated Key Exchange}), are resistant to the
password compromise scenario described above, \emph{even when the
client is communicating directly with a malicious impersonator.}
Furthermore, these protocols alert the client to the presence of an
impersonator, allowing the SSH user to curtail further communications
in high-risk situations.  However, existing SPAKA protocols impose
requirements that make them difficult to deploy in most settings,
especially when legacy servers and legacy hashed-password files are
involved (see Section~\ref{sec:related}).

In this paper, we present the first password-based authentication and
key establishment protocol to satisfy, in the context of SSH, the
three design principles listed above.  We show that the secure
password storage and the secure key establishment requirements can be
achieved by comparing the authentication credentials of the user and
the server using \emph{secure function evaluation} (SFE)~\cite{yao82},
in a legacy-compatible manner.

The main insight that enables backward compatibility with existing
infrastructures is that SFE gives the protocol complete flexibility to
compute arbitrary hash functions while performing authentication.
This makes our protocol suitable as a ``drop-in'' authentication
module in most legacy environments, requiring only that the server and
client software be updated to use the new protocol.


\vspace{1ex}
\noindent
\textbf{Organization of the paper.}  In Section~\ref{sec:related}, we
discuss related work, and explain why existing SPAKA protocols are not
suitable for SSH in terms of the three requirements previously
listed. In Section~\ref{sec:overview}, we present a technical overview
of our problem setting, as well as our proposed solution. In
Section~\ref{sec:proto}, we describe our contributions in further
detail, and in Section~\ref{sec:eval} we evaluate our implementation.
