
\section*{Appendix}
\label{sec:appA}

\subsection*{EKE}

\emph{Encrypted Key Exchange} (EKE) was developed by Bellovin and
Merritt to allow two parties to communicate using a weak secret, such
as a low-entropy password~\cite{bellovin92}. Suppose $A$ and $B$ share
such a secret, $P$, and that $A$ wishes to securely communicate
message $M$ to $B$. Na\"ively, $A$ could use a symmetric cipher $E$
with $P$ as the key, and send $E_P(M)$ to $B$; however, because $P$ is
weak, this is subject to brute-force attack by an adversary who wishes
to learn $M$. Instead, EKE uses a temporary asymmetric key pair to
exchange a stronger shared symmetric key to use for the duration of
the session. The steps of the protocol, simplified for clarity of
presentation, are as follows:
\begin{enumerate}
\item $A$ generates a random asymmetric key pair $(E_A, D_A)$ and
  sends the public key encrypted with the weak password, $E_P(E_A)$,
  to $B$.
\item $B$ generates a random symmetric key $R$, and sends
  $E_P(E_A(R))$ to $A$.
\item $A$ uses $S$ and $D_A$ to decrypt $R$, which is used as a strong
  session key from this point on.
\end{enumerate}
In this protocol, the weak secret $P$ is used only to encrypt $E_A$
and $E_A(R)$. Thus, it is critical that $E_A$ and $E_A(R)$ are
essentially random, to thwart brute-force attacks on the keyspace of
$R$.

Central to the correct execution of this protocol is that both $A$ and
$B$ have knowledge of $P$, and can use it to perform symmetric
encryption. However, this violates our first and second assumptions,
as current servers do not typically store user passwords in clear text
for a number of reasons~\cite{lamport81}.

\subsection*{AEKE}

\emph{Augmented Encrypted Key Exchange} (AEKE) was developed by
Bellovin and Merritt to relax the constraint in EKE that the server
possess the shared secret $P$ in clear text~\cite{bellovin93}. Rather,
in AEKE, it is assumed that the server possesses knowledge of a secure
hash of the password, $H(P)$. The first part of AEKE proceeds exactly
as in EKE, with the exception that all uses of $P$ in the protocol are
replaced with uses of $H(P)$. However, $B$ must be able to verify that
$A$ possesses the true shared secret $P$, and not merely its hash. To
allow this, they conceptualize a new one-way function $F(P,R)$, where
$R$ is the strong session key as in EKE, as well as a predicate
$T(H(P),F(P,R),R)$. By requiring that $T$ evaluates to \emph{true} if
and only if $H(P)$ and $F(P,K)$ are computed using the same shared
secret $P$, they provide a mechanism for $B$ to verify that $A$ knows
$P$. Thus, two additional steps are needed:
\begin{enumerate}
  \item[4.] $A$ sends $R(F(P,R))$ to $B$. This proves $A$'s ownership
  of the original shared secret $P$, as $H(P)$ cannot be used to
  calculate $F(P,R)$.
  \item[5.] $B$ decrypts $A$'s message to obtain $F(P,R)$ and accepts
  the identity of $A$ only if the $T$ predicate evaluates to
  \emph{true}.
\end{enumerate}
Bellovin and Merritt proposed one practical scheme for instantiating
$H$, $F$, and $T$ on real systems. By using the public key in a
digital signature scheme to define $H(P)$, $A$ can compute $F(P,R)$ by
signing $R$ with the corresponding private key. Evaluating $T$ then
amounts to verifying the signature sent by $A$ with $H(P)$. However,
it is improbable that the hashing methods in use on current servers
can be used in this manner, so this algorithm fails to meet our first
requirement.

\subsection*{$\Omega$-Method}

The $\Omega$-method was developed by Gentry \textit{et al.} as a way
of hardening arbitrary password-based key exchange (PAKE) protocols
against the event that the server is compromised~\cite{gentry06}. In
other words, the $\Omega$-method allows any PAKE protocol
$\mathcal{P}$ to operate under the assumption that the server stores
only a hash of the user's password, $H(P)$. It works as follows:
\begin{enumerate}
\item $A$ and $B$ run $\mathcal{P}$ as they would normally,
  substituting every occurrence of $P$ with $H(P)$. After running
  $\mathcal{P}$, both parties possess a shared secret key $R$.
\item $B$ derives a second session key $K'$ from $K$, and uses it to
  send an encrypted secret key $D_A$ from an asymmetric key pair
  $(D_A, F_A)$, to $A$: $E_{K'}(E_{P}(D_A))$
\item $A$ receives $E_{K'}(E_{P}(D_A))$, derives $K'$ from $K$ in a
  manner identical to $B$, and decrypts using $K'$ and $P$ to obtain
  $D_A$. $A$ then signs a transcript of the protocol using the secret
  key, and sends it to $B$:
  $\mathsf{Sign}_{D_A}(\mathsf{transcript})$. The client then derives
  the final session key $K''$ from $K$, to be used for the remainder
  of the session.
\item $B$ receives $\sigma =
  \mathsf{Sign}_{D_A}(\mathsf{transcript})$, and uses $F_A$ to verify
  it: $\mathsf{Verify}_{F_A}(\mathsf{transcript},$ $\sigma)$. If it
  checks out, then $B$ knows that $A$ possesses $P$, and derives the
  shared session key $K''$ to continue.
\end{enumerate}
The basic idea of the $\Omega$-method can be summarized as follows:
$\mathcal{P}$ is run as normal, substituting $H(P)$ for $P$ as
necessary, but afterward, $A$ must prove knowledge of $P$ such that
$H(P)$ is in agreement with the $B$'s authentication records. However,
the manner in which the method performs this requires the server to
store a non-trivial amount of additional information in the form of an
asymmetric key pair, so it fails to meet our first requirement.

\subsection*{Multiple-Server}

Ford and Kaliski~\cite{ford00} also consider the problem of using a
weak password to bootstrap a secure means of communication, and the
risk associated with storing some form of the password database on the
server. Their approach consists of protecting passwords from server
compromise by distributing trust across multiple servers, thereby
eliminating the chance of total password compromise by infiltration of
a single server. In their scheme, the client's weak password is used
to derive a strong secret by means of multiple \emph{hardening
servers}, and combines the result of the individual interactions into
a strong secret that can be used for further communication. It works
as follows:
\begin{enumerate}
\item For each of one or more servers $B_1, \ldots, B_n$, $A$ runs a
  \emph{hardening protocol} to obtain a hardened password $R_i$ based
  on the original weak password $P$. None of the servers learns either
  the original password, or the hardened password.
\item Using the set of hardened passwords $R_i, 0 \le i \le n$, $A$
  derives a strong secret $K_i$ that is used to authenticate with
  $B_i$. Additional strong secrets $K_{n+1}, \ldots, K_m$ can be
  derived in a similar fashion, and used for future communications.
\end{enumerate}
Ford and Kaliski point out that \emph{all} of the servers must
collaborate to determine whether $A$ has given the true password, and
no strict subset of the servers alone can determine the additional
strong secrets $K_{n+1}, \ldots, K_m$. They presented a hardening
protocol based on discrete logarithm cryptography, and claim that it
can be extended to elliptic curve cryptography~\cite{ford00}.

Because of their reliance on multiple servers for basic security, it
is clear that the approach of Ford and Kaliski~\cite{ford00}, as well
as later approaches that are similar in
nature~\cite{brainard03,mackenzie02}, fail to meet our first
requirement of straightforward backwards compatibility with existing
infrastructures.
