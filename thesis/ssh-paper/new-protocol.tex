\documentclass[12pt]{article}

\usepackage{fullpage}
\usepackage{times}

%new environments
\newtheorem{theorem}{Theorem}
\newtheorem{lemma}[theorem]{Lemma}
\newtheorem{corollary}[theorem]{Corollary}
\newtheorem{definition}{Definition}
\newtheorem{fact}{Fact}
\newtheorem{example}{Example}

\begin{document}

We present a protocol for our context, which has better detection
protocol than the generic construction presented by Aumann and
Lindell~\cite{}.  Recall that in our context there are two parties in
the protocol: party 1 (client) has input $x$ and party 2 (server) has
input $y$. They want to jointly compute the functionality $(x,y)
\longmapsto ( \delta (H(x) = y),\delta (H(x) = y))$ where
$\delta_{H(x) = y}$ is equal to $1$ if $H(x) = y$; otherwise it is
$0$. If client gets output of $1$, it means that client was
authenticated by the server. In our context, the reader should
interpret $x$ as the password and $y$ as the hash of the password (in
other words client can only successfully authenticate if they know a
password whose hash matches what the server has). The key idea is that
the hash function $H$ is included in the functionality, which makes
our protocol resilient against weakness suggested in
Section~\ref{}. Our protocol description assumes that the reader is
familiar with the basics of secure-function evaluation (such as
garbled circuit construction and oblivious transfer).

\begin{itemize}

\item {\bf (Step 1)}  Client creates $l$ garbled circuits $C_1,\cdots,C_l$ of the  for $\delta (H(x), y)$. Let server's 
input $y \; = \; y_1 \cdots y_m$ be $m$ bits. The wire keys corresponding
to the $j$-bit of server's input for the $i$-th garbled circuit $C_i$
is denoted by $k^0_{i,j}$ and $k^1_{i,j}$. Client sends circuits
$C_1,C_2,\cdots,C_l$ to the server.

\item {\bf (Step 2)} Client and server execute the $OT_1^2$ protocol $m$ times. In the $j$-th instance of $OT_1^2$ the
client acts as a sender with inputs $k^0_{1,j} \| k^0_{2,j} \| \cdots
\| k^0_{l,j}$ and $k^1_{1,j} \| k^1_{2,j} \| \cdots \| k^1_{l,j}$ and
the server acts the chooser with input $y_j$ (the $j$-th bit of the
input). Notice that concatenating the keys prevents the server from
learning keys corresponding to different bits, e.g., server cannot
learn keys $k^0_{1,j}$ and $k^1_{2,j}$.

\item {\bf (Step 3)} Server chooses a random set $S \subseteq \{ 1,2, \cdots , l \}$ and sends $S$ to the client.

\item {\bf (Step 4)} Client reveals wire keys for circuits $C_j$ such that $j \in S$ to the server (we call this step opening the
circuits $C_j$ such that $j \in S$). Client also provides wire keys for its input $x$ for circuits $C_j$, $j \not\in S$.

\item {\bf (Step 5)} If the circuits $C_j$ ($j \in S$) are not well-formed (the circuits do not compute $\delta (H(x),y)$ or the 
keys are not consistent with what was sent in step 2), the server
sends $0$ to the client. Server computes $C_j$ ($j \not\in S$) and
obtains answers $o_j$ ($j \not\in S$). Server sends $\bigwedge_{j
\not\in S} o_j$ to the client.



\end{itemize}

It is clear that if both client and server are honest, then the client
will successfully authenticate to the server if and only if it has a
password $x$ whose hash $H(x)$ is equal to the input of the server
$y$. Some of the important features of our protocol are:
\begin{itemize}
\item Note that server learns the wire keys corresponding to {\it one input}.
In other words, it is not possible for the server to evaluate circuit $C_i$
on input $x_1$ and circuit $C_j$ ($j \not= i$) on a different input $x_2$.
This is rationale behind concatenating the wire keys in step 2 of the protocol.

\item Suppose the server denies client's authentication request. Client does
not know whether it was denied authentication because it was trying to cheat (by
sending invalid circuits) or it had the wrong password $x$.

\item Assume that out of the $l$ garbled circuits $C_1, \cdots , C_l$ the
circuits with index $j \in B$ (where $B \subseteq \{ 1,2, \cdots ,
l\}$) are not valid. The only way client's cheating is not detected is
if the random set $S$ chosen by the server in step 3 is the complement
of $B$. Since $S$ is chosen randomly, the probability of this event is
$2^{-l}$.
\end{itemize}

Next we formally argue about the privacy of the client and the server.

\noindent
{\bf Client's privacy:} Assume that the client is honest and the
server is controlled by an adversary ${\cal A}$.  ${\cal A}$'s view
consists of the $l$ garbled circuits $C_1,\cdots,C_l$, messages
received during the $m$ $OT_1^2$ protocols, all keys for $C_j$ ($j \in
S )$, where $S \subseteq \{ 1,2, \cdots, l \}$ is chosen by ${\cal
A}$), and keys corresponding to the client's input $x$ for circuits
$C_j$ ($j \not\in S$). Assume that views corresponding to the $m$
$OT_1^2$ protocols only reveal the secrets corresponding to the input
$y$ of ${\cal A}$ (let the ${\cal A}$ input be $y = y_1 \cdots y_m$
then the server learns $k^{y_k}_{j,k}$ $1 \leq j \leq l$ and $1 \leq k
\leq m$). This follows from the privacy of the underlying
oblivious-transfer protocol. For example, if one uses the Naor-Pinkas
oblivious-transfer protocol, then we have the information-theoretic security
for the server. Assuming that the encryption scheme used
to construct the garbled circuits is semantically secure, revealing
the wire keys for circuits $C_j$ $(j \in S)$ does not reveal any
information about the client's input. Consider the circuits $C_j$ $(j
\not\in S$). Server can evaluate this circuit on $(x,y)$ but learns
nothing else. Consider an ensemble of garbled circuits $C'_j$ $(j
\not\in S)$, 
where $C'_j$ computes the constant function $\delta (H(x),y)$.\footnote{ We assume $C'_j$ is constructed from the same
encryption scheme that was used to construct $C_1,\cdots,C_j$.}. If
the encryption-scheme is semantically secure, then ${\cal A}$ cannot
distinguish between circuits $C_j$ and $C'_j$ (for $j \not\in
S$). Essentially ${\cal A}$ only learns whether hash of client's
password is equal to its input and nothing else.

\noindent
{\bf Server's privacy:} Assume that the server is honest and the
client is malicious. The client is controlled by an
adversary ${\cal A}$. We construct a simulator ${\sf Sim}$ which works
in the ideal model. ${\sf Sim}$ acts as the server for ${\cal A}$.
\begin{itemize}
\item ${\cal A}$ sends $l$ copies of the garbled circuits $C_1,\cdots,C_l$ to ${\sf Sim}$.
\item ${\sf Sim}$ acts as TP for ${\cal A}$ for the $m$ oblivious transfer protocols. ${\sf Sim}$ knows the
inputs of ${\cal A}$ (which in the case of the honest client's are the wire keys corresponding to the 
server's inputs).
\item ${\sf Sim}$ chooses a random set $S_1 \subseteq \{ 1,2, \cdots, l \}$ and sends it to ${\cal A}$.

\item ${\cal A}$ sends all the wire keys corresponding to circuits $C_j$ $(j \in S_1)$.

\item ${\sf Sim}$ rewinds ${\cal A}$, and sends the complement of $S_1$ to ${\cal A}$. 
${\cal A}$ sends all the wire keys corresponding to circuits $C_j$ $(j \not\in S_1)$. Note that after this
step ${\sf Sim}$ knows the wire keys corresponding to all the garbled circuits $C_1,\cdots,C_l$.

\item ${\sf Sim}$ rewinds ${\cal A}$, picks a random set $S \subseteq \{ 1,2,\cdots, l \}$, and
sends it to ${\cal A}$. ${\cal A}$ sends wire keys corresponding to the circuits $C_j$ $(j \in S)$. 

\item ${\cal A}$ provides the wire keys for all circuits $C_j$ ($j \not\in S)$. Note that since ${\sf Sim}$
knows wire keys for all the garbled circuits, it can now construct ${\cal A}$'s input $x$.

\item ${\sf Sim}$ checks the validity of all the circuits. If any of the circuits is found to be invalid,
${\sf Sim}$ sends $0$ to ${\cal A}$. Otherwise $x$ is send to the TP and output of the TP is sent to
${\cal A}$. ${\sf Sim}$ outputs whatever ${\cal A}$ does.
\end{itemize}
Let $E_1$ be the event that a honest server detects cheating in the real
model, and $E_2$ be the event that ${\sf Sim}$ detects cheating in the
ideal model. It is easy to see that if $E_1$ and $E_2$ are true, then
the view of ${\cal A}$ in the real model is indistinguishable from the
view of ${\sf Sim}$ in the ideal model. The probability of event $E_1
\wedge E_2$ not happening is bounded by $2^{-l+1}$. Hence we have
security for the server in the covert model with $\epsilon =
2^{-l+1}$.

\noindent
{\bf Replay attacks:}



\end{document}

