\section{Related Work}
\label{sec:related}

\begin{figure*}[t]
  \centering

  \begin{tabular}{l|ccc}
    & \textit{Legacy compatibility} & \textit{Secure password storage} & \textit{Mutual authentication} \\
    \hline
    EKE & \textit{\textbf{X}} & \textit{\textbf{X}} & \checkmark \\
    AEKE & \textit{\textbf{X}} & \checkmark & \checkmark \\
    $\Omega$-Method & \textit{\textbf{X}} & \checkmark & \checkmark \\
    Multiple-Server & \textit{\textbf{X}} & \checkmark & \checkmark
  \end{tabular}
  
  \caption{A comparison of existing SPAKA protocols. The protocols
    listed are EKE~\cite{bellovin92}, AEKE~\cite{bellovin93},
    $\Omega$-method~\cite{gentry06}, and
    Multiple-Server~\cite{ford00}. There are a number of protocols in
    the literature similar in nature to EKE and AEKE; these are
    referenced in the text but left out of this table for the sake of
    clarity.}
  \label{tab:proto}
%%  \vspace{-2.5em}
\end{figure*}

\textit{Secure Password and Key Authentication} (SPAKA) is a class of
authentication protocols designed to guarantee confidentiality of
secrets even against active malicious adversaries. There have been
several SPAKA protocols proposed in the literature with varying
properties. The first such protocol was described by Bellovin and
Merritt~\cite{bellovin92}. This protocol, \emph{Encrypted Key
Exchange} (EKE), was designed to allow two parties to communicate using
a weak secret, such as an \emph{easily memorable} password. The
authors observed that a standard symmetric cryptosystem keyed on the
weak secret does not provide strong security, and instead proposed the
use of a temporary asymmetric key pair to exchange a stronger shared
symmetric key to use for the full duration of the session. The key
element that makes this secure is that the bit strings which represent
keys in several asymmetric schemes are essentially random, and so
difficult to verify in a brute-force attack on the key exchange
messages. Thus, this protocol provides a way of accomplishing basic
mutual authentication, and satisfies requirement \textit{(3)} from our
list. However, it does not satisfy \textit{(1)} or \textit{(2)}, as
common deployments store only a hash of the client's password on the
server. A number of subsequent protocols have the same properties in
terms of our
requirements~\cite{abdalla06,abdalla05,bellare00,boyko00,gentry05,mack00,zhang04},
including a recent scheme by Abdalla \textit{et al.} proved secure in
the universal composability model~\cite{abdalla08}, and one by Katz
\textit{et al.} in the standard model~\cite{katz01,katz02}, which was
later extended to additional cryptographic assumptions by Gennaro and
Lindell~\cite{gennaro08,gennaro03}.

Bellovin and Merritt developed \emph{Augmented Encrypted Key Exchange}
(AEKE) \cite{bellovin93} to address these shortcomings by relaxing the
requirement that the server possess knowledge of the client's password
in clear text. Rather, they allow the server to possess only a hash of
the password, which preserves the secrecy of the client's password in
a scenario where the password database is compromised by an
adversary. To prevent impersonation of the client by such an
adversary, the protocol uses primitives that allow one party to verify
that the other has knowledge of both the password and its hash, and
proposed two schemes for selecting these primitives. The first scheme
uses a class of \emph{commutative one-way hash functions}. However,
there are no known families of commutative hash functions that possess
the information-hiding properties required to guarantee the security
of the protocol, making this scheme of theoretical interest only. The
second scheme defines the hashed password stored by the server as the
public key in a digital signature scheme. Then, to prove knowledge of
the original password, the client signs the session key with the
corresponding private key. For our purposes, AEKE implemented with
this scheme satisfies requirements \textit{(2)} and \textit{(3)} from
our list. However, correctly storing the public key on the server may
require substantial changes to infrastructure, and violates
requirement \textit{(1)}.

More recently, Gentry \textit{et al.}~\cite{gentry06} proposed the
$\Omega$-method for converting an arbitrary PAKE protocol that is
\emph{not} resilient to server compromise into one that is secure in
such a scenario. They proved the security of their method in the
universal composability framework. However, all known feasible
implementations of their method require the server to store additional
information, namely a public/private key pair with the secret key
encrypted. Thus, applying this method to one of the previously
described protocols will result in a set of implementation constraints
basically equivalent to AEKE~\cite{bellovin93}, and ultimately fail to
satisfy requirement \textit{(1)}.

Ford and Kaliski~\cite{ford00} presented a SPAKA protocol that
protects the secrecy of the client's password against server
compromise by distributing it among many servers. When
the client authenticates, it interacts with each server to establish a
set of strong secrets, after which the servers collaborate to validate
the client's identity. A number of subsequent protocols adopt this
basic functionality; MacKenzie \textit{et al.} generalize the protocol
to a threshold setting~\cite{mackenzie02}, and Brainard \textit{et
  al.} present a lightweight protocol that reduces the computation
load on the client~\cite{brainard03}. While these protocols satisfy
our security requirements (\textit{(2)} and \textit{(3)}), they have
the obvious drawback of requiring a specific server-side architecture
that may not be common in many settings, and thus fail to satisfy
requirement \textit{(1)}.

Further details on each type of protocol described in this
section, as well as specific explanations of why they are not suited
for the case of SSH authentication, are discussed in the appendix.
