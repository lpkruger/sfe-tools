\begin{flushleft}
Assume that both parties' inputs include the $OBDD(f)$ for the Boolean
function $f(x_{1},x_{2},\cdots,x_{n})$ with the ordering $x_{1}<x_{2}<\cdots<x_{n}$.
Furthermore, Alice holds the inputs $(i_{1},\ldots,i_{k})$ corresponding
to the first $k$ variables $x_{1},\ldots,x_{k}$, and Bob has the
inputs $(i_{k+1},\ldots,i_{n})$.
\par\end{flushleft}
\begin{enumerate}
\item Alice performs the following steps: 

\begin{enumerate}
\item She traverses the $OBDD(f)$ using her input $(i_{1},\cdots,i_{k})$,
which results in a node $v_{init}$ at level $k$.
\item She uniformly and independently at random creates $(n-k)$ pairs of
secrets $(s_{1}^{0},s_{1}^{1}),\cdots,(s_{n-k}^{0},s_{n-k}^{1})$.
In addition, for each node $v$ in the $OBDD(f)$ whose level is between
$k$ and $n-1$, Alice also creates a secret $s_{v}$.
\item She assigns a uniformly random label to each node whose level is between
$k$ and $n$. We refer to the randomly assigned label of node $v$
using the notation $label(v)$.
\item Next, Alice augments $OBDD(f)$ with some number of dummy nodes (to
ensure that Bob always traverses $n-k$ nodes in his phase of the
protocol).
\item Alice garbles all nodes whose level is between $k$ and $n-1$ in
the following manner. Let $v$ be a node in $OBDD(f)$ such $k\leq{\it level}(v)\leq n-1$
and define ${\it level}(v)=\ell$. The encryption of node $v$, denoted
by $E^{(v)}$, is a label and a randomly ordered ciphertext pair \[
\left(label(v)\,\,,\,\, E_{s_{v}\oplus s_{\ell-k+1}^{0}}(label(low(v))\,\|\, s_{{\it low}(v)})\,\,\,,\,\,\, E_{s_{v}\oplus s_{\ell-k+1}^{1}}(label(high(v))\,\|\, s_{{\it high}(v)})\right)\,\,\,,\]
 where the labels are pre-pended to the secret with a separator symbol
and the order of the ciphertexts is determined by a fair coin flip.
Roughly speaking, the secrets corresponding to the $0$-successor
and $1$-successor of node $v$ are encrypted with the secret corresponding
to $v$ and its level.


Note that dummy nodes have the same structure as normal nodes, except
that the ciphertext pair contain encryptions of the same message since
dummy nodes have the same $0$ and $1$-successors. Provided the encryption
scheme is semantically secure, this poses no problem since the keys
are chosen uniformly at random.

Lastly, there are two terminal nodes of the form $(b,label(t_{b}))$
for $b=0$ or $1$. Recall that $OBDD(f)$ has two terminal nodes,
denoted as $0$ and $1$, that are at level $n$.

\item Once Alice is done encrypting, she sends to Bob the encryption of
all nodes whose level is between $k$ and $n$ and the secret $s_{v_{init}}$
corresponding to node $v_{init}$ at level $k$. We called this the
garbled OBDD.
\end{enumerate}
\item Bob performs the following steps: 

\begin{enumerate}
\item He engages in $n-k$ 1-out-of-2 oblivious transfers to obtain the
secrets corresponding to his input. For example, if his input $i_{j}$
is $0$, then he obtains the (level) secret $s_{j-k}^{0}$; otherwise,
he obtains the secret $s_{j-k}^{1}$.
\item Now Bob is ready to start his computation. Suppose $i_{k+1}=0$. With
$s_{1}^{0}$ and $s_{v_{init}}$, he decrypts both ciphertexts in
$E^{(v_{init})}$ and decides which gives the correct result by using
the verifiable range property of the encryption scheme. Bob now has
both $s_{{\it low}(v)}$ (the secret corresponding to the $0$-successor
of $v_{init}$) and $label(low(v))$ (which tells Bob which encrypted
node is used to evaluate his next input). Continuing this way, Bob
eventually obtains a label corresponding to one of the terminal nodes,
which determines the result of the OBDD on the shared inputs. Bob
sends this result to Alice. 
\end{enumerate}
\end{enumerate}

