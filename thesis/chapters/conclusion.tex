\chapter{Conclusions}
In this thesis, we have looked at techniques for making secure function
evaluation practical. In chapter 4, we looked at the use of Ordered
Binary Decision Diagrams (OBDDs) to optimize the evaluation of certain
kinds of functions. Our conclusions were that OBDDs can provide a
significant performance gain for certain functions. In chapter 5,
we discussed a specialized use of SFE to solve the classic problem
of password authentication in a novel way. The solution presented
has useful properties that no existing solution has, and is fast enough
for practical use. Chapter 6 presented general techniques for applying
SFE to dynamic programming problems. The techniques presented are
of general algorithmic interest, but we applied these techniques to
traditional genomic algorithms and showed that efficient protocols
for computing edit distance and Smith-Waterman alignment scores can
be constructed. Chapter 7 presented another efficient protocol to
solve the K-means clustering problem, used in genomics and other machine
learning problems. In chapter 8, we delved into the cryptographic
primitives used in SFE and showed how modular square roots could be
used as an efficient trapdoor function for $1$ out of $N$ oblivious
transfer.

Our thesis statement was that SFE can be used to solve practical problems
today. We believe that this dissertation demonstrates by example that
SFE is ready to move from a curiosity of purely theoretical interest
into a technique that is ready for practical application today on
the internet.
