
\chapter{Background and Cryptographic Primitives}


\section{Threat Models\label{sub:Threat-Models}}

It is well known that there are many threats and malicious parties
on the Internet, as discussed in the previous section. We now present
formalizations of these threats into threat models from the literature,
suitable for formal cryptographic analysis. The three threat models
presented here, known as the {}``semi-honest'', {}``malicious'',
and {}``covert'' models, cover a variety of assumptions and levels
of trust among cooperating and non-cooperating parties, ranging from
parties who basically trust each other but want to keep information
private, all the way to parties who have no trust relationship at
all. In particular, the {}``malicious'' model is strong enough to
encompass malicious behavior from any party in a protocol, as will
be discussed below.

In the {}``semi-honest'' threat model, also known as {}``honest
but curious'', or {}``passive'', \cite{GMW87} a party to the computation
is assumed to behave correctly and follow the protocol as prescribed.
However, the party also runs additional probabilistic polynomially
bounded computation on the side in order to learn information to which
he is not entitled. A security proof using the semi-honest threat
model implies that the protocol as designed does not {}``leak''
information. The semi-honest model is an important theoretical tool
despite the fact that it is weaker than the malicious model and does
not capture the full range of malicious behaviors we would expect
of an adversary. It is useful because a protocol that has been proven
secure in the semi-honest model is guaranteed to have an analogous
protocol that is secure in the malicious model \cite{GMW87}. The
proof of this theorem is constructive and involves extending the protocol
in an automated way, using a protocol {}``compiler'', to a more
secure protocol. Essentially, the protocol compiler inserts additional
steps into the protocol to force the parties to prove to one another
their faithful adherence to the protocol. This fact notwithstanding,
the semi-honest model may itself be a realistic model in certain cases,
for example, when the parties communicating need to preserve privacy
of data from one another, but also have a pre-existing trust relationship
sufficient to believe the other party will not maliciously attempt
to cheat the protocol.

In the {}``malicious'' threat model, an adversarial party is free
to use any available methods to thwart the computation, including
sending false or inconsistent messages at any step of the protocol.
The malicious threat model naturally characterizes the malicious behavior
that a secure protocol would need to protect against. If a protocol
is shown to be secure in the malicious threat model, then we can reasonably
assume that a malicious party or interloper will not be able to learn
any private information by attacking the protocol.

In the {}``covert'' threat model, similar to the {}``malicious''
model, the adversary is also free to use any available methods to
thwart the computation. Although the allowable behavior of the adversary
is the same as in the malicious model, the security guarantee is relaxed.
In particular, a protocol is considered secure in the covert model
if the probability of the attacker not getting caught is small but
non-negligible. The covert model can be considered appropriate for
real-world scenarios only when the consequences to the adversary for
being caught are a significant deterrence to trying to cheat (\ie
loss of reputation, monetary or legal penalties, etc) compared to
the benefit of a successful attack. \cite{aumannlindell}


\section{Primitives \label{sub:Primitives}}

With these threat models firmly in mind, we can begin examining methods
for constructing protocols that are secure in one or more of the threat
models. We find in the design of secure protocols that certain sub-protocols
and techniques are used so frequently as building blocks that they
deserve to be called primitives. These primitives solve simple problems,
such as one party needing to choose a value from another party in
a privacy preserving way, and also are self-contained cryptographic
steps that cannot be decomposed into still simpler steps.


\subsection{Oblivious Transfer}

Oblivious transfer is a protocol originally proposed by Rabin \cite{Rabin81}.
Informally, a 1-out-of-n oblivious transfer, denoted as $OT_{n}^{1}$,
is a protocol between two parties, the \emph{Chooser} and the \emph{Sender}.
The Sender's inputs into the protocol are $n$ values $v_{1},...,v_{n}$.
The Chooser's input is an index $i$ such that $1\le i\le n$. As
a result of the protocol, the Chooser receives $v_{i}$, but does
not learn any additional information about the rest of the Sender's
values. The Sender learns nothing.

There are a variety of OT protocols presented in cryptographic literature.
We focus on the Naor-Pinkas OT protocol \cite{NaorPinkas99}, based
on discrete logarithms, because it is a very efficient OT protocol
that we have chosen for practical implementations. The performance
characteristics of this protocol are discussed in chapter \ref{sub:Comparison-with-Naor-Pinkas}.
The steps of the Naor-Pinkas OT protocol are summarized next. We will
also investigate other OT methods based on modular square roots in
chapter \ref{sec:OT-SquareRoots}
\begin{enumerate}
\item Let $q$ be a prime number and let $g$ be a generator for the field
$\mathbb{Z}_{q}$. $q$ and $g$ are known to both parties. Also,
let $s\in\left\{ 0,1\right\} $ be the chooser's secret choice, and
let $M_{0}$ and $M_{1}$ be the sender's messages. At the end of
the protocol, the chooser should learn $M_{s}$ only and no party
should learn any other information.
\item The sender picks a random number $C\in\mathbb{Z}_{q}$ and sends $C$
to the chooser\emph{.}
\item The chooser picks a random number $k\in\mathbb{Z}_{q}$ and computes
two values: $K_{0}$ and $K_{1}$, where $K_{s}=g^{k}$ and $K_{1-s}=\frac{C}{g^{k}}$.
The chooser sends $K_{0}$ to the sender.
\item The sender encrypts $M_{0}$ and $M_{1}$. Specifically, the sender
chooses random values $r_{0}$ and $r_{1}$ and computes $E_{0}=E_{K_{0}^{r_{0}}}\left(M_{0}\right)$
and $E_{1}=E_{K_{1}^{r_{1}}}\left(M_{1}\right)$ where $E_{k}\left(M\right)$
is a conventional encryption function. Ideally $E_{k}\left(M\right)=H\left(k\right)\oplus M$
where $H\left(x\right)$ is a random oracle, and in practice is typically
implemented by a secure hash function. The values $g^{r_{0}}$, $E_{0}$,
$g^{r_{1}}$ and $E_{1}$ are sent to the chooser.
\item The chooser can decrypt message $E_{s}=E_{K_{s}^{r_{s}}}\left(M_{s}\right)$
by computing the key $K_{s}^{r_{s}}=\left(g^{r_{s}}\right)^{k}$.
Thus the chooser learns $M_{s}$. The chooser cannot decrypt $E_{1-s}$
unless the chooser can find $k'$ such that $g^{k'}=\frac{C}{g^{k}}$.
Thus, the security of Naor-Pinkas OT depends on the Diffie-Hellman
assumption that finding discrete logarithms is a hard problem.
\end{enumerate}

\subsection{Homomorphic Encryption}

Homomorphic Encryption is a class of public key encryption algorithms
that satisfies a homomorphism property. If $E(x)$ is an encryption
function for a cipher, then an additive homomorphic cipher satisfies
$E(a+b)=E(a)\oplus E(b)$ where $\oplus$ is an efficiently computable
operator that requires no secret information. Similarly, a multiplicative
homomorphic cipher satisfies $E(ab)=E(a)\otimes E(b)$. Some of the
most famous public key ciphers have the multiplicative homomorphic
property, including the Elgamal cipher \cite{elgamal85} and the RSA
cipher \cite{rivest83rsa}. The homomorphic properties have traditionally
been considered undesirable for general purpose cryptography \cite{jmsw02}.
%
\begin{comment}
mention Cramer-Shoup? 
\end{comment}
{}Specifically, the malleability of ciphertexts can allow the adversary
to violate integrity constraints, and also make such ciphers insecure
against %
\begin{comment}
because the homomorphic structure aids in cryptanalysis and allows
encrypted messages to be modified, violating integrity constraints.
\cite{jmsw02}. This leads to insecurity against 
\end{comment}
{}\emph{adaptive chosen ciphertext} (CCA2) attacks \cite{bleichenbacher98chosen}.
However, despite these concerns, homomorphic encryption schemes have
found use in novel cryptographic applications such as secure voting
\cite{benaloh94}. 

The homomorphic encryption methods we use in secure function evaluation
must satisfy some properties:
\begin{description}
\item [{Semantically-secure~cipher}] A cipher is \textit{semantically
secure} \cite{Goldwasser:Micali} if it satisfies certain properties.
Let $(G,E,D,M)$ be a public-key encryption scheme. $E_{e}(m)$ and
$D_{d}(c)$ are the encryption and decryption functions for plaintext
$m$ and ciphertext $c$, with respect to a public/private key pair
($e,d)$. $G$ is a key generation function that can be used to randomly
generate $(e,d)$ pairs, and $M$ is the message space respectively. \end{description}
\begin{itemize}
\item Informally, the encryption scheme is semantically secure if the ciphertext
leaks no useful information about the plaintext even if the attacker
has previously observed many plaintext-ciphertext pairs on plaintexts
of his choice. 
\item Formally, the encryption scheme is semantically secure if for every
probabilistic polynomial time algorithm $A$ there exists a probabilistic
polynomial-time algorithm $A'$ such that for every probabilistic
ensemble $\left\{ X_{n}\right\} _{n\in\mathbb{N}}$ with $\mid X_{n}\mid \le\mbox{poly}\left(n\right)$
every pair of polynomially bounded functions $f,h:\left\{ 0,1\right\} ^{*}\rightarrow\left\{ 0,1\right\} ^{*}$,
every positive polynomial $p$ and all sufficiently large $n$, then
the following inequality holds:
\end{itemize}
\[
\Pr\left[A\left(1^{n},E_{G_{1}\left(1^{n}\right)}\left(X_{n}\right),1^{\mid X_{n}\mid },h\left(1^{n},X_{n}\right)\right)=f\left(1^{n},X_{n}\right)\right]\]


\[
<\Pr\left[A'\left(1^{n},1^{\mid X_{n}\mid },h\left(1^{n},X_{n}\right)\right)=f\left(1^{n},X_{n}\right)\right]+\frac{1}{p(n)}\]

\begin{itemize}
\item With any semantically secure encryption scheme, encrypting the same
message twice will yield different ciphertexts with high probability,
so $E(m)$ must be a randomized one-to-many function representing
a set of possible ciphertexts that can be obtained by encrypting $m$.
Naturally, if $m_{1}\neq m_{2}$, then $E(m_{1})\cap E(m_{2})=\emptyset$ \end{itemize}
\begin{description}
\item [{Homomorphic~Cipher}] A cipher is called \emph{homomorphic} if
there is a computable homomorphism between plaintexts and ciphertexts.
There exist various homomorphic ciphers with homomorphisms over addition,
multiplication, and other operators. An \emph{additive homomorphic
cipher}, which will be used in this thesis, satisfies the following
homomorphism:\end{description}
\begin{itemize}
\item There exists a computable function $f$, computable without the private
key or other secret information, such that for all messages $m_{1}$,
$m_{2}$, and $c_{1}\in E(m_{1})$, $c_{2}\in E(m_{2})$, the following
property holds:\\
 $f\left(c_{1},c_{2}\right)\in E(m_{1}+m_{2})$ 
\item There exists a computable function $g$ such that for all $m_{1}\in M$
and $\alpha\in M$, $c_{1}\in E(m)$ implies that $g(c_{1},\alpha)\in E(\alpha m_{1})$.
In addition, $g$ must be computable without using the private key
or other secret information. This property follows automatically from
the previous requirement, because it is always possible to define
$g$ in terms of $O(\log\alpha)$ invocations of the function $f$. 
\end{itemize}
There are several encryption schemes that satisfy these properties,
of which Paillier's encryption scheme, based on composite residue
classes, is widely used for its efficiency \cite{Paillier99}. In
the Paillier cryptosystem, the message space is all $m<n$, where
$n=pq$ for $p$ and $q$ prime. The ciphertext space is $E(m)<n^{2}$.
Let $g<n^{2}$ such that $g$ has order $n\alpha$. Using the public
key $(g,n)$, the encryption function $E(m)=g^{m}r^{n}\left(\mbox{mod }n^{2}\right)$,
for a random $r<n$. Using the private key $\lambda=\mbox{lcm}(p-1,q-1)$,
the decryption function for ciphertext $c$ is $m=\frac{L\left(c^{\lambda}\mbox{ mod }n^{2}\right)}{L\left(g^{\lambda}\mbox{ mod }n^{2}\right)}\mbox{ mod }n$
where $L(u)=\frac{u-1}{n}$ is a well-defined function for $u\equiv1\;(\mbox{mod }n)$.
Notice that $E(m_{1})\cdot E(m_{2})=g^{m_{1}}r_{1}^{n}g^{m_{2}}r_{2}^{n}=g^{m_{1}+m_{2}}(r_{1}r_{2})^{n}\in E(m_{1}+m_{2})$,
which satisfies the additive homomorphic property. Further details
can be found in \cite{Paillier99}.


\section{Secure Function Evaluation}

One of the fundamental cryptographic primitives for designing privacy-preserving
protocols is \textit{secure function evaluation (SFE)}. Techniques
for optimizing SFE are the topic of this thesis. A protocol for SFE
enables two parties $A$ and $B$ with inputs $x$ and $y$ respectively
to jointly compute a function $f(x,y)$ while preserving the privacy
of the two parties' respective inputs. At the end of the protocol,
party $A$ only knows its input $x$ and the value of the function
$f(x,y)$, and a similar condition holds for $B$. It was proven by
Yao \cite{Yao86} and Goldreich, Micali, and Wigderson \cite{GMW87}
that for any polynomially-computable function $f$, there exist protocols
for securely evaluating $f$ that execute in polynomial time. Both
proofs are constructive, and provide a method for transforming a Boolean
circuit description of the function $f$ into a protocol for secure
evaluation of $f$. These protocols are summarized below.


\subsection{Garbled Circuit Method \label{sub:Garbled-Circuit-Method}}

Consider any Boolean circuit $C$, and two parties: Alice and Bob,
who wish to evaluate $C$ on their respective inputs $x$ and $y$.
In Yao's {}``garbled circuits'' method \cite{Yao86}, Alice transforms
the circuit in such a way that Bob can evaluate it obliviously, i.e.,
without learning Alice's inputs or the values on any internal circuit
wire except the output wires. The steps are as follows: 
\begin{enumerate}
\item Alice generates two random keys $k_{i,0}$ and $k_{i,1}$ for each
circuit wire $i$, one representing $0$ on that wire, the other representing
$1$. For all wires in the circuit except input wires, the truth table
for the corresponding Boolean gate is encrypted. If $g(x,y)$ is a
gate with input wires $j$ and $l$, and output wire $i$, then the
truth table value for $g(x,y)$ is encoded as $E_{k_{j,x}}\left(E_{k_{l,y}}\left(k_{i,g(x,y)}\right)\right)$.
Here, $k_{j,x}$ is the encryption key for value $x$ of wire $j$,
and similarly for $k_{l,j}$. $k_{i,g(x,y)}$ is the encryption key
for the output wire of $g$ with value $g(x,y)$ The four encrypted
values representing $g(0,0)$, $g(0,1)$, $g(1,0)$, and $g(1,1)$
fully specify the gate $g$. Alice sends the garbled circuit to Bob.
Computation of the garbled circuit does not depend on input values
and can be performed in advance. However, the same garbled circuit
must not be used more than once, or Alice's privacy may be violated. 
\item Alice sends the keys corresponding to her own input wires to Bob.
Bob obtains the keys corresponding to his input wires from Alice using
an $OT_{2}^{1}$ protocol. For each of Bob's input wires, Bob acts
as the chooser using his corresponding input bit to the function as
the choice into $OT_{2}^{1}$ , and Alice acts as the sender with
the two wire keys for that wire as her inputs into $OT_{2}^{1}$ . 
\item Bob evaluates the circuit. Because of the way that the garbled circuit
is constructed, Bob, having one wire key for each gate input, can
decrypt exactly one row of the garbled truth table and obtain the
key encoding the value of the output wire. Yao's protocol maintains
the invariant that for every circuit wire, Bob learns exactly one
wire key. Because wire keys are random and the mapping from wire keys
to values is not known to Bob (except for the wire keys corresponding
to his own inputs), this does not leak any information about actual
wire values. 
\end{enumerate}
After these steps the circuit has been evaluated obliviously by Bob.
The final step is for Bob to send to Alice her output wire keys, from
which she will learn Alice's designated outputs. A complete description
of Yao's method and security proofs can be found in \cite{Goldreich:vol2}.
A simple example of an encrypted AND gate is shown here:

Suppose Alice has $x$ and Bob has $y$ and Alice and Bob wish to
compute $z=x\wedge y$. The truth table for this gate is as follows:

\begin{tabular}{|c|c|c|}
\hline 
$x$ & $y$ & $z=x\wedge y$\tabularnewline
\hline
\hline 
$0$ & $0$ & $0$\tabularnewline
\hline 
$0$ & $1$ & $0$\tabularnewline
\hline 
$1$ & $0$ & $0$\tabularnewline
\hline 
$1$ & $1$ & $1$\tabularnewline
\hline
\end{tabular}

Alice generates a pair of random wire keys for each of $x$, $y$,
and $z$. Then, the encrypted AND gate will look like this:

\begin{tabular}{|c|c|c|}
\hline 
$K_{x}$ & $K_{y}$ & $E\left(K_{z}\right)$\tabularnewline
\hline
\hline 
$k_{x,0}$ & $k_{y,0}$ & $E_{k_{x,0}}\left(E_{k_{y,0}}\left(k_{z,0}\right)\right)$\tabularnewline
\hline 
$k_{x,0}$ & $k_{y,1}$ & $E_{k_{x,0}}\left(E_{k_{y,1}}\left(k_{z,0}\right)\right)$\tabularnewline
\hline 
$k_{x,1}$ & $k_{y,0}$ & $E_{k_{x,1}}\left(E_{k_{y,0}}\left(k_{z,0}\right)\right)$\tabularnewline
\hline 
$k_{x,1}$ & $k_{y,1}$ & $E_{k_{x,1}}\left(E_{k_{y,1}}\left(k_{z,1}\right)\right)$\tabularnewline
\hline
\end{tabular}

For the sake of concreteness, suppose $x=1$ and $y=0$. Then Alice
will send $k_{x,1}$to Bob, and Bob will choose $k_{y,0}$ in the
$OT_{2}^{1}$ step, and Bob will only be able to successfully decrypt
$E_{k_{x,1}}\left(E_{k_{y,0}}\left(k_{z,0}\right)\right)$ from the
truth table.


\subsection{Secure Computation With Random Shares}

\cite{GMW87} presents a protocol for securely evaluating circuits
known as \emph{secure computation with shares} (SCWS). This protocol
maintains the invariant that, for every circuit wire $w$, Alice learns
a random value $s$ and Bob learns $b_{w}\oplus s$, where $b_{w}$
is the bit value of the wire. Therefore, Alice's and Bob's shares
add up to $b_{w}$, but because the shares are random, neither party
knows the actual wire value. For each output wire of the circuit,
Alice and Bob combine their shares to reconstruct the circuit output.
Suppose $g(x,y)$ is a 2-input gate, $x$ and $y$ are the input wires
to the gate $g$ and $x_{a}\oplus x_{b}=x$ and $y_{a}\oplus y_{b}=y$
are Alice and Bob's shares of $x$ and $y$. The following steps will
securely evaluate the gate: 
\begin{enumerate}
\item Alice selects a random bit $z_{a}\in\left\{ 0,1\right\} $ 
\item Alice constructs a quadruple. The quadruple contains each of the 4
possible values of $g\oplus a_{z}$\\
$\left(g(x_{a},y_{a})\oplus z_{a},\, g(x_{a},1-y_{a})\oplus z_{a},\, g(1-x_{a},y_{a})\oplus z_{a},\, g(1-x_{a},1-y_{a})\oplus z_{a}\right)$. 
\item Using an $OT_{4}^{1}$ protocol, Bob selects the bit from Alice's
quadruple with index $s=2x_{b}+y_{b}$, in other words, the position
corresponding to Bob's shares of wires $x$ and $y$. The value received
by Bob is $z_{b}=g(x_{a}\oplus x_{b},y_{a}\oplus y_{b})\oplus z_{a}=g(x,y)\oplus z_{a}$,
which is Bob's share of the output wire of gate $g$. Alice's share
is $z_{a}$.
\item As a shortcut, certain gates $g$ can be evaluated without interaction
between Alice and Bob. For example, if $g=x\oplus y$ then $g_{a}=x_{a}\oplus y_{a}$
and $g_{b}=x_{b}\oplus y_{b}$ maintains the share invariant without
communication between Alice and Bob.
\end{enumerate}
At the beginning of the evaluation, Alice sets her share of the input
wires to her input values, and her share of Bob's input wires to $0$,
and vice versa for Bob. Each gate $g$ may be evaluated after Alice
and Bob have computed their shares of the gate's input wires. Thus,
by repeated applying the above steps, the entire circuit can be evaluated
starting from the inputs, and progressing gate by gate until the output.
Further details and security proofs are presented in \cite{Goldreich:vol2}.

In practice, Yao's garbled circuit method is more commonly used, because
it is more efficient than the computation with shares. The garbled
circuit method requires only a single transfer of data from Alice
to Bob, followed by an $OT_{2}^{1}$ for each the $|B|$ values representing
Bob's inputs. These OTs can be combined into a single parallel OT.
Then Bob obliviously evaluates the entire circuit on his own, and
sends the output keys of Alice's outputs back to her. In contrast,
the SCWS method requires an $OT_{4}^{1}$ for each gate. Some of these
OTs can be parallelized, but because of dependencies not all can be
parallelized, and therefore the protocol requires at least $depth(C)$
distinct rounds of the OT, where $depth(C)$ is the maximum number
of gates along any path from an input to an output. The increased
number of OTs, combined with the increased number of rounds needed
to execute them, makes the SCWS evaluation protocol primarily of theoretical
interest. However, the SCWS principle can be emulated with the Yao
protocol, by explicitly including extra gates in the circuit to combine
and split the share values. We will use this technique later in this
thesis.


\section{Implementations}

In recent years, there have been implementations of SFE undertaken
by researchers to design secure multiparty protocols. In the past,
SFE was considered a theoretical topic too expensive for practical
use, but the convergence of ubiquitous communication using the Internet,
more efficient cryptographic primitives, and the exponentially increasing
availability of processing power and network bandwidth are making
SFE an area of increasingly significant practical value.


\subsection{Fairplay}

Fairplay \cite{Fairplay} is an example of an SFE implementation designed
to enable wider application of SFE. Fairplay is the first system,
designed to be practical, that attempts to make SFE using Yao's protocol
available to a wider audience. It consists of a compiler that takes
as input a function $f$ defined using a procedural language called
\emph{Secure Function Description Language} (SFDL), and outputs a
Boolean circuit to evaluate $f$ using a description language called
\emph{Secure Hardware Description Language} (SHDL). Fairplay also
includes an implementation of the two party Yao protocol which securely
evaluates an SHDL function. \cite{Fairplay} provides the first empirical
measurements from an implementation of the Yao protocol.


\subsection{Application specific}

Fairplay showed that the classic protocol for SFE is still quite expensive
for all but the simplest circuits. There has been much research effort
in designing more efficient privacy-preserving protocols for many
problems of interest. In \cite{Reiter:CCS:2003}, a compiler was implemented
for generating secure two-party protocols for a restricted class of
functions built from modular arithmetic. The particular design was
motivated by the desire to build efficient secure protocols such as
signature schemes and threshold cryptography. Secure protocols have
been implemented for many problems such as auctions \cite{NPS99},
set intersection \cite{FNP04}, and conducting surveys \cite{FNP04}.
A particularly important application of secure computation is discussed
in the next section.


\section{Privacy Preserving Data Mining }

Initial focus in this area was on construction of decision trees from
distributed data sets \cite{Agrawal-Srikant,Lindell-Pinkas}. There
is also a significant body of research on privacy-preserving mining
of association rules \cite{Gehrke:2002,RizviHarista,VaidyaClifton:2002}.
In general, there are two approaches for designing privacy-preserving
data mining algorithms. The first approach is to use transformations
to perturb the data set before the algorithm is applied, by replacing
sensitive data with random unique identifiers. This approach for designing
privacy-preserving algorithms is taken by several researchers \cite{Klusch,MeruguGhosh,Oliveira}.
However, this approach suffers from the lack of formal security guarantee,
and has been shown to be vulnerable to data correlation attacks \cite{Malin04},
which is why we use secure multiparty computation in this thesis,
the basis of the other approach. A survey of such techniques is presented
in \cite{PinkasCryptoPPDM02}. 


%
\begin{comment}
\bibliographystyle{plain} \bibliographystyle{plain}
\bibliography{privacy,somesh,crypto}

\end{comment}
{} 
