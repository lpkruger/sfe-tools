\section{Implementation and Experiments}
\label{sec:experiments}

In this section we describe various components of our
implementation (called SFE-OBDD), which is based on protocol 2 described
in Section~\ref{subsec:protocol2}. We also present experimental results comparing the
performance of our implementation against Fairplay. 
The following major conclusions can be drawn from our experimental
investigation:
\begin{itemize}
\item The restriction operation used in protocol 2 can significantly
reduce the size of the OBDD, which can lead to reduced bandwidth while
executing the protocol. 

\item Our OBDD protocol outperforms Fairplay circuit in terms of
bandwidth in most functions in our benchmark.  However, some functions which
have inefficient OBDD representations can perform far worse.

\item Execution times of our implementation and Fairplay are dominated
by the oblivious-transfer protocol. Since the oblivious-transfer
component of our protocol and the protocol used by Fairplay is the
same, with respect to execution times we did not observe as much
improvement as in bandwidth.  

\item Converting an OBDD into a circuit  results in a 
blowup in size. We implemented a reverse compiler that takes an OBDD and
converts it into a circuit description, which can be used in Fairplay.
Typically this conversion from OBDD to circuit resulted in a
blowup in size by a factor of $5-10$, depending on post-conversion
optimizations.  However, if the original circuit is particularly inefficient, 
it is possible for some gain to be achieved due to the canonical 
representation property of OBDDs.

\end{itemize}

\subsection{Implementation}
Our implementation consists of the following components:

\begin{enumerate}
\item An implementation of  protocol 2
as described in Section~\ref{subsec:protocol2}.

\item Fairplay uses the {\it secure hardware definition language (SHDL)} to
describe circuits. Because we wanted to compare the performance of our
protocol with the Yao circuit protocol on identical functions, 
we implemented an OBDD compiler 
that takes as input a file
describing a function in SHDL, and produces the corresponding OBDD.
Note that both the SHDL used by Fairplay and the BDD representation
originate from the same high level SFDL description, this means that
the OBDD and circuit are evaluating the same functions.
\end{enumerate}

For the cryptographic primitives we use exactly the same
implementation as Fairplay. We use the 1-out-of-2 $(OT^{2}_{1})$
proposed by Noar and Pinkas~\cite{Noar-Pinkas:2001}, and described in section
\ref{sub:Oblivious-Transfer}, the encryption
function is $E_k (m)$ was ${\rm SHA-1}(k) \oplus (m \,\|\,0^n)$.


Our OBDD compiler allows us to directly compare the efficiency of our
implementation to Fairplay. The OBDD compiler takes as input a file
containing an SHDL description and produces a file containing the
description of the corresponding OBDD. This file can then be used as
an input to the SFE protocol. Our compiler uses the
JavaBDD~\cite{JavaBDD}, and BuDDy~\cite{BuDDy} libraries, which
provide functions to construct and manipulate OBDDs. It is well known
that the size of an OBDD can be sensitive to the ordering of
variables~\cite{Bryant:BDD}. In some cases, variable ordering can make
the difference between a OBDD that is linear versus exponential in the
number of variables. Our SHDL to OBDD compiler allows the user to
specify a particular variable ordering, which is useful if the user
has domain knowledge about the function. If this is not practical, the
compiler includes an optimizer that attempts to automatically find a
variable ordering that yields an efficient OBDD, making use of
heuristic functions built into the BuDDy library. Although in general
finding the optimal variable ordering is
NP-hard~\cite{Bollig-Wegener:1996}, we have found that in practice the
optimizer can find good orderings for various functions we considered.


\subsection{Experimental Results}

We compared our implementation to classic Yao circuits as implemented in
Fairplay, discussed in section~\ref{sub:Fairplay}.
We used various functions, some of which are included in the Fairplay
distribution, as test
cases to perform a comparison of the Fairplay protocol with our
OBDD-based SFE protocol. The description of the functions are given
in Figure~\ref{fig:descriptions}.  Each function was evaluated
at several word sizes to evaluate scalability.

\begin{figure*}
\begin{tabular}{|c|p{6in}|}
\hline 
And&
This is a circuit that computes the bitwise AND of two N bit numbers.
Alice has N inputs, Bob has N inputs, and there are N bits of output.\tabularnewline
\hline 
Add&
This is a circuit that computes the addition of two N bit numbers.
Alice has N inputs, Bob has N inputs, and there are N bits of output.
The high bit is discarded.\tabularnewline
\hline 
Eq&
This is a simple equality comparator of two N-bit numbers.  There is
one bit of output. \tabularnewline
\hline
Mul&
This is a circuit that computes the unsigned multiplication of two
N bit numbers. Alice has N inputs, Bob has N inputs, and there are
N bits of output. The output is modulo $2^N$.  
We were unable to test N=16 because of exponential blowup in the BDD
\tabularnewline
\hline 
KDS&
This is a circuit that implements a simply-keyed database lookup.
Alice supplies N key/value pairs, and Bob supplies a key. The output
is the value of that key, or 0 if the key is not found. The keys are
${\tt log}_2(N)$ bits, and the data are 24 bits. 
We were unable to test N=16 because of exponential blowup in the BDD.
\tabularnewline
\hline 
Mil&
This is the millionaire's problem. Alice and Bob each have and N bit integer
as inputs, and there is 1 bit of output indicating if Alice's input is larger than Bob's.\tabularnewline
\hline 
Parity&
Alice and Bob each have N-bits of input. They want to jointly compute the
parity of their combined input bits.  There is one bit of output.\tabularnewline
\hline
\end{tabular}
\caption{Description of the functions used in our experiments.  Each function
was tested with N=4, N=8, and N=16 except where indicated}
\label{fig:descriptions}
\end{figure*}


For each function, Figure~\ref{table:size} shows the sizes of the
OBDDs and the corresponding circuit used by Fairplay. The size of an
OBDD is the number of vertices in it. The size of a Fairplay circuit
with $n_1$ gates of arity $1$, $n_2$ gates of arity $2$, and $n_3$
gates of arity $3$ was computed as $2 \times n_1 + 4 \times n_2 + 8
\times n_3$ (this represents the number of entries in the truth table
for the circuit).  For the OBDDs, we show the sizes of the original
OBDDs (in column marked as {\sf Original}), with the dummy nodes added
(in column marked as {\sf Full}), and after Alice has performed the
restriction operation on OBDDs with dummy nodes (in column marked as
{\sf Res}). Recall that dummy nodes are added so that regardless of
Alice's inputs Bob has to follow the same number of edges. In
protocol~2 Alice computes the OBDD for restriction on its input of the
function to be jointly computed for the variable. These operations are
described in detail in Section~\ref{sec:sfe-obdd}. Two observations
can be made from Figure~\ref{table:size}.
\begin{itemize}
\item Restriction can significantly reduce the size of
the OBDD. For example, for the function Mil16  restriction
reduces the size of the OBDD by more than half. 

\item Notice that for all functions except MUL8, MUL16, KDS4, KDS8, and KDS16 the
size of the OBDD after restriction is smaller than the size of the
circuit used in Fairplay. This suggests the choice of when to use our
system over Fairplay depends on the function to be computed. 
\end{itemize}




%
\begin{figure}


\begin{center}\begin{tabular}{|c||c|c|c||c|}
\hline 
&
\multicolumn{3}{c||}{BDD}&
FairPlay\tabularnewline
&
{\sf Original} &
{\sf Full} &
{\sf Res} &
\tabularnewline
\hline
\hline 

Add4 & 32 & 40 & 22 & 56 \tabularnewline \hline
Add8 & 72 & 96 & 54 & 128 \tabularnewline \hline
Add16 & 152 & 208 & 118 & 272 \tabularnewline \hline
And4 & 14 & 18 & 10 & 24 \tabularnewline \hline
And8 & 26 & 34 & 18 & 48 \tabularnewline \hline
And16 & 50 & 66 & 34 & 96 \tabularnewline \hline
Eq4 & 18 & 24 & 11 & 102 \tabularnewline \hline
Eq8 & 27 & 41 & 18 & 230 \tabularnewline \hline
Eq16 & 51 & 81 & 34 & 486 \tabularnewline \hline
KDS4 & 416 & 578 & 466 & 356 \tabularnewline \hline
KDS8 & 4084 & 7149 & 6283 & 780 \tabularnewline \hline
KDS16 & * & * & * & 2244 \tabularnewline \hline
MUL4 & 54 & 75 & 28 & 114 \tabularnewline \hline
MUL8 & 1685 & 1800 & 1087 & 586 \tabularnewline \hline
MUL16 & * & * & * & 2682 \tabularnewline \hline
Mil4 & 24 & 34 & 22 & 52 \tabularnewline \hline
Mil8 & 46 & 70 & 40 & 116 \tabularnewline \hline
Mil16 & 94 & 150 & 90 & 244 \tabularnewline \hline
parity4 & 18 & 18 & 10 & 30 \tabularnewline \hline
parity8 & 34 & 34 & 18 & 62 \tabularnewline \hline
parity16 & 66 & 66 & 34 & 126 \tabularnewline \hline

\hline
\end{tabular}\end{center}
\caption{\label{table:size}Size of the OBDDs and the circuit used in Fairplay for
functions shown in Figure~\ref{fig:descriptions}.  Values labeled "*" could
not be converted to OBDDs because of exponential blowup.}

\end{figure}




Our experimental results were obtained using a pair of machines
connected on a local $100$-megabit network.  The machines were
configured with 3.0Ghz Intel Pentium4 processors, 1 gigabyte of
memory, and the Centos Linux 4.0 operating system using a modified
Linux 2.6.9 kernel.  For each function shown in
Figure~\ref{fig:descriptions} we executed our OBDD-based and Fairplay
code on a Sun Microsystems Java 1.5.0\_04 JVM. Alice was run on one
machine, and Bob on the other.  For each execution, we measured the
network bandwidth used (number of bytes transferred between Alice and
Bob) and the execution time. The number reported for each trial is the
average of three trials.  Figure~\ref{table:size-bandwidth} shows the
size of the garbled OBDD and garbled circuit in bytes and the network
bandwidth for our implementation and Fairplay. Recall that the
garbled OBDD is the structure that Alice sends to Bob to
evaluate. With respect to network bandwidth our implementation
outperformed Fairplay for seven out of the nine functions. We have
implemented a reverse compiler that takes as input an OBDD, and
outputs an SHDL description of a Boolean circuit to evaluate the OBDD.
This is performed via a straightforward transformation that takes each
node in the OBDD and produces corresponding 3-input MUX gate in the
Boolean circuit.  Then, an optimization pass is run using the same
techniques described in~\cite{Fairplay}.  The column labeled as
``Converted Fairplay'' in Figure~\ref{table:size-bandwidth} shows the size
of the encrypted circuits produced by running the FairPlay protocol on
the converted BDDs.  Note that in a few cases, the converted circuit
is actually more efficient than the corresponding Fairplay circuit.  This occurs
because FairPlay is not guaranteed to produce an optimal circuit from
the function description. However, it is clear that our protocol that
directly uses OBDD is much more efficient than the protocol produced
by the reverse compiler.


Figures~\ref{fig:OBDD-timing}
and~\ref{fig:Fairplay-timing} show the execution times for SFE-OBDD and Fairplay.  The elapsed
execution times (EET) are shown in the last column.  Columns 2-5 show
the breakdown by sub-task, which are IPCG (initializations, parsing,
and garbling), CC (circuit communication, Alice sending the garbled
structure to Bob), OT (Oblivious Transfer, Bob obtaining secrets
corresponding to its input),
and EV (circuit evaluation, Bob evaluating the garbled structure). These sub-tasks were also used by the
Fairplay paper~\cite{Fairplay}. In general, because the time for OT dominates
the execution time, we only observe moderate improvement in SFE-BDD over Fairplay
for execution times. 

%
\begin{figure*}


\begin{center}\begin{tabular}{|c||c|c|c||c|c|}
\hline 
Function & \multicolumn{3}{c|}{\sf Size in bytes.} & \multicolumn{2}{c|}{\sf Bandwidth in bytes} \\ 
\hline 
	&  SFE-OBDD & Fairplay & Converted Fairplay & SFE-OBDD & Fairplay\\ \hline

Add4 & 970 & 1915 & 5382 & 3604 & 4684 \tabularnewline \hline
Add8 & 1979 & 4214 & 11408 & 6590 & 9000 \tabularnewline \hline
Add16 & 3992 & 8821 & 23442 & 12557 & 17645 \tabularnewline \hline
And4 & 739 & 1080 & 1866 & 3373 & 3849 \tabularnewline \hline
And8 & 1206 & 2153 & 3140 & 5813 & 6938 \tabularnewline \hline
And16 & 2134 & 4299 & 5546 & 10696 & 13117 \tabularnewline \hline
Eq4 & 582 & 2977 & 2690 & 3214 & 5716 \tabularnewline \hline
Eq8 & 828 & 6527 & 3918 & 5434 & 11240 \tabularnewline \hline
Eq16 & 1324 & 13626 & 7012 & 9892 & 22295 \tabularnewline \hline
KDS4 & 14966 & 12248 & 65946 & 51219 & 13984 \tabularnewline \hline
KDS8 & 185282 & 25608 & 682333 & 261077 & 27838 \tabularnewline \hline
MUL4 & 1108 & 3286 & 8354 & 3739 & 6052 \tabularnewline \hline
MUL8 & 32206 & 15706 & 288317 & 36814 & 20493 \tabularnewline \hline
Mil4 & 892 & 1662 & 4278 & 3524 & 4399 \tabularnewline \hline
Mil8 & 1400 & 3542 & 8248 & 6012 & 8256 \tabularnewline \hline
Mil16 & 2790 & 7306 & 16859 & 11356 & 15972 \tabularnewline \hline
parity4 & 577 & 1092 & 3237 & 3209 & 3830 \tabularnewline \hline
parity8 & 828 & 2181 & 6172 & 5438 & 6897 \tabularnewline \hline
parity16 & 1324 & 4359 & 11983 & 9889 & 13033 \tabularnewline \hline

\end{tabular}\end{center}

\caption{\label{table:size-bandwidth}Size in bytes of the garbled OBDD, garbled circuit, and garbled circuit using the
reverse compiler. Network bandwidth in  bytes.}

\end{figure*}

\begin{figure}
\begin{center}
\begin{tabular}{|c|c|c|c|c|c|} \hline
Fn & IPCG &      CC &    OT &      Eval &     EET  \\ \hline

Add4 & 13.75\% & 5.62\% & 79.69\% & 0.94\% & 0.32 \\ \hline
Add8 & 13.08\% & 3.80\% & 82.07\% & 1.05\% & 0.47 \\ \hline
Add16 & 11.39\% & 2.36\% & 84.42\% & 1.83\% & 0.76 \\ \hline
And4 & 12.42\% & 5.73\% & 81.21\% & 0.64\% & 0.31 \\ \hline
And8 & 9.38\% & 4.02\% & 85.94\% & 0.67\% & 0.45 \\ \hline
And16 & 7.44\% & 2.75\% & 89.39\% & 0.41\% & 0.73 \\ \hline
Eq4 & 12.66\% & 5.38\% & 81.33\% & 0.63\% & 0.32 \\ \hline
Eq8 & 9.33\% & 3.90\% & 86.12\% & 0.65\% & 0.46 \\ \hline
Eq16 & 8.97\% & 2.48\% & 88.14\% & 0.41\% & 0.72 \\ \hline
KDS4 & 4.79\% & 0.95\% & 93.86\% & 0.40\% & 2.52 \\ \hline
KDS8 & 10.91\% & 1.69\% & 87.13\% & 0.27\% & 5.50 \\ \hline
MUL4 & 14.77\% & 5.23\% & 79.38\% & 0.62\% & 0.33 \\ \hline
MUL8 & 31.80\% & 4.11\% & 63.45\% & 0.63\% & 0.63 \\ \hline
Mil4 & 13.44\% & 5.62\% & 80.00\% & 0.94\% & 0.32 \\ \hline
Mil8 & 11.37\% & 3.86\% & 84.12\% & 0.64\% & 0.47 \\ \hline
Mil16 & 10.55\% & 3.03\% & 85.88\% & 0.53\% & 0.76 \\ \hline
parity4 & 10.67\% & 4.78\% & 83.71\% & 0.84\% & 0.36 \\ \hline
parity8 & 9.37\% & 3.92\% & 86.06\% & 0.65\% & 0.46 \\ \hline
parity16 & 8.55\% & 2.62\% & 88.41\% & 0.41\% & 0.72 \\ \hline

\end{tabular}
\end{center}
\caption{Elapsed execution time (EET) in seconds and their breakdowns into sub-tasks for
SFE-OBDD.}
\label{fig:OBDD-timing}
\end{figure}

\begin{figure}
\begin{center}
\begin{tabular}{|c|c|c|c|c|c|} \hline
Fn & IPCG &     CC &     OT &     Eval &     EET \\ \hline

Add4 & 17.65\% & 19.00\% & 63.12\% & 0.23\% & 0.44 \\ \hline
Add8 & 16.13\% & 15.96\% & 67.38\% & 0.53\% & 0.56 \\ \hline
Add16 & 10.74\% & 9.67\% & 79.12\% & 0.48\% & 0.84 \\ \hline
And4 & 11.92\% & 20.44\% & 67.40\% & 0.24\% & 0.41 \\ \hline
And8 & 11.78\% & 16.07\% & 71.96\% & 0.19\% & 0.54 \\ \hline
And16 & 9.78\% & 6.35\% & 83.48\% & 0.38\% & 0.79 \\ \hline
Eq4 & 19.51\% & 11.85\% & 68.15\% & 0.49\% & 0.41 \\ \hline
Eq8 & 15.44\% & 16.52\% & 67.68\% & 0.36\% & 0.56 \\ \hline
Eq16 & 13.23\% & 9.84\% & 76.70\% & 0.23\% & 0.85 \\ \hline
KDS4 & 33.33\% & 12.75\% & 53.33\% & 0.58\% & 0.34 \\ \hline
KDS8 & 35.98\% & 11.92\% & 51.43\% & 0.66\% & 0.45 \\ \hline
MUL4 & 21.89\% & 2.16\% & 75.41\% & 0.54\% & 0.37 \\ \hline
MUL8 & 21.75\% & 7.99\% & 69.89\% & 0.37\% & 0.54 \\ \hline
Mil4 & 38.27\% & 8.26\% & 53.28\% & 0.19\% & 0.53 \\ \hline
Mil8 & 16.98\% & 9.25\% & 73.40\% & 0.38\% & 0.53 \\ \hline
Mil16 & 18.78\% & 9.01\% & 71.99\% & 0.22\% & 0.92 \\ \hline
parity4 & 20.78\% & 11.25\% & 67.73\% & 0.24\% & 0.41 \\ \hline
parity8 & 49.55\% & 0.39\% & 49.94\% & 0.13\% & 0.77 \\ \hline
parity16 & 14.63\% & 1.15\% & 83.97\% & 0.25\% & 0.79 \\ \hline


\end{tabular}
\end{center}
\caption{Elapsed execution time (EET) in seconds and their breakdowns into sub-tasks for
Fairplay.}
\label{fig:Fairplay-timing}
\end{figure}

