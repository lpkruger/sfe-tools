\begin{proof} 

%[{\it Proof of Claim~\ref{claim:protocol1secure}}]

Intuitively, Bob's security follows directly from the security of the
1-out-of-2 oblivious transfer protocol he uses to obtain the secrets
corresponding to his input. Alice's security follows from both the
security of the oblivious transfer protocol (allowing Bob to only
obtain only one key per node) and the semantic security of the
encryption scheme (which allows Bob to only decrypt one entry in each
node). We now flesh out the details by providing a simulation proof
from Alice and Bob's view of the protocol. Let $x$ and $y$ be the
inputs of Alice and Bob respectively and $\Pi$ be a protocol for
secure-function evaluation of $f(x,y)$.  Let ${\sf VIEW}_A^\Pi (x,y)$
and ${\sf VIEW}_B^\Pi (x,y)$ be the view of Alice and Bob for the run
of the protocol $\Pi$ on input $x$ and $y$ (view of a party consists
of its input, output, and all messages it receives during the
execution of the protocol). In a simulation proof one needs to show
two probabilistic polynomial-time algorithms $S_A$ and $S_B$ such that
$S_A (x,f(x,y))$ and $S_B (y,f(x,y))$ are computationally
indistinguishable from ${\sf VIEW}_A^\Pi (x,y)$ and ${\sf VIEW}_B^\Pi
(x,y)$, respectively.  For a precise definition of a simulation proof
the reader should refer to~\cite[Chapter 7]{Goldreich:vol2}.

We first consider the case where Alice is corrupt. Alice's view in an
execution of Protocol 1 consists of her view of the oblivious transfer
protocol executions and the output of the function from Bob at the
end. We now build a simulator that simulates Alice's view given access
only to her input and output. Because the oblivious transfer protocol
is secure, there exists a simulator that can simulate the transcript
of Alice's view of the oblivious transfer protocol without knowing
Bob's input. On input $(i_1,\ldots,i_k, f(i_1,\ldots,i_n))$, the
simulator first simulates Alice's view of all $n-k$ executions of the
oblivious transfer protocol by repeatedly running the oblivious
transfer protocol simulator. Using a standard hybrid argument on the
transcripts of all $n-k$ executions oblivious transfer protocols, we
see that if the oblivious transfer protocol is secure, then the
distributions of the simulated and real combined transcripts of all
$n-k$ oblivious transfer executions are indistinguishable with
non-negligible probability.

Finally, the simulator writes $f(i_1,\ldots,i_n)$ on the transcript
of Alice's view. We now show that the distribution of the output $f$
is indistinguishable (except with negligible probability) from the
real output, which amounts to showing that Bob outputs
$f(i_1,\ldots,i_n)$ correctly on a real interaction. By the security
of the oblivious transfer protocol, Bob is provided with the correct
keys corresponding to its input during each execution of the oblivious
transfer protocol. Applying claim~\ref{claim:protocol1correct}, it
follows immediately that, except with negligible probability, Alice
obtains the correct output from Bob, except with negligible
probability. Therefore, the distribution of the simulated transcript
is indistinguishable, except with negligible probability, from a real
transcript, concluding the case when Alice is corrupt.

We now consider the case when Bob is corrupt. Given
$(i_{k+1},\ldots,i_n, f(i_1,\ldots,i_n))$, the simulator $\simu$ must
simulate both a garbled OBDD that Bob can use to correctly compute
$f(i_1,\ldots,i_n)$, and Bob's view of the $n-k$ executions of the
oblivious transfer protocol. We first show how $\simu$ simulates the
$n-k$ oblivious transfer protocol executions. As in the previous case,
because the oblivious transfer protocol is secure, there exists a
simulator that can simulate the transcript of Bob's view of the
oblivious transfer protocol without knowing Alice's input. Therefore,
$\simu$ simulates Bob's view of all $n-k$ executions of the
oblivious transfer protocol by running the oblivious transfer protocol
simulator $n-k$ times. Using a standard hybrid argument on the
transcripts of the oblivious transfer protocols, we see that if the
oblivious transfer protocol is secure, then the distributions of the
simulated and real transcripts of all $n-k$ oblivious transfer
executions are indistinguishable except with negligible probability.

We now show how $\simu$ builds a garbled OBDD that Bob can use to
successfully compute $f(i_1,\ldots,i_n)$. Since $\simu$ does not know
$i_1,\ldots,i_k$, it cannot generate the garbled OBDD according to the
protocol instructions. Instead, $\simu$ generates a garbled OBDD that
always evaluates to $f(i_1,\ldots,i_n)$ regardless of the keys
used. Such a garbled OBDD is built by first generating a chain of $n-k$
garbled nodes $n_{k+1},\ldots,n_{n}$ such that Bob's computation
starts at  $n_{k+1}$ and proceeds along the chain through
$n_{k+2}$ and so on, before ending at node $n_{n}$; note that there is
one such node for every level from $k+1$ to $n$. To ensure the
computation always proceeds along this chain, \textit{both}
ciphertexts in garbled nodes $n_{k+1},\ldots,n_{n-1}$ are encryptions
(under different keys) of the same label-key message such that the
label points to the next node along the chain and the node key
combined with the level key allows successful decryption of that node;
for example, simulated node $n_j$ for $k+1 \leq j\leq n-1$ has the
form
\[
\begin{array}{l}
\left( label(n_j)\,\,,\,\,E_{s_{n_j} \oplus s_{l}^0} (label(n_{j+1}) \,\|\, s_{n_{j+1}}) \right. \\
\left. E_{s_{n_j} \oplus s_{l}^1} (label(n_{j+1}) \,\|\, s_{n_{j+1}})\right)\,\,\,.
\end{array}
\]
Node $n_{n}$ is the terminal node and it is set to
$f(i_1,\ldots,i_n)$. Once $n_{k+1},\ldots,n_{n}$ is generated, the
simulator generates a number of ``fake'' nodes so that the simulated
garbled OBDD contains the correct number of nodes; this number can be
determined from $OBDD(f)$. Fake nodes are nodes whose ciphertext pair
contain encryptions under different keys of the same label-key
message; in a fake node, the label, the keys used to encrypt the
ciphertext pair, and the label-key message encrypted in the ciphertext
pair are chosen randomly.

All that remains is to show that the distribution of the simulated
garbled OBDD is indistinguishable from that of a real garbled OBDD. We
do this by using a standard hybrid argument over the nodes in the
garbled OBDD. Specifically, we run hybrid experiments with garbled OBDDs
where real nodes are replaced by simulated nodes. We define the hybrid
distributions such that $H_0(i_1,\ldots,i_n)$ contains the real
garbled OBDD and $H_{B}(i_1,\ldots,i_n)$ contains the simulated garbled
OBDD where $B$ is the number of non-dummy nodes in the real garbled
OBDD. We do not need to consider dummy nodes in our hybrid experiment
OBDDs because dummy nodes have the same distribution as the simulated
fake nodes and do not affect our argument.

We now define the hybrid garbled OBDD in experiment
$H_i(i_1,\ldots,i_n)$; the difficulty here is that the hybrid OBDD
contains both real and simulated nodes but must still allow Bob to
correctly compute $f(i_1,\ldots,i_n)$. First, we traverse the real
garbled OBDD and label a node as active if it is used by Bob in the
process of evaluating the OBDD and inactive otherwise. Note that there
will be only $n-k$ active nodes. Next, we order the nodes in the
garbled OBDD by their level with level $j+1$ nodes placed ahead of
level $j+2$ and so on; within the same level, nodes are ordered
arbitrarily. The hybrid OBDD is defined as follows: first take the real
garbled OBDD and replace the first $i$ non-dummy nodes as follows:
inactive nodes are replaced with simulated fake nodes. An active node
at level $j$ is altered by replacing its current ciphertext pair with
two encryptions of the label-key message corresponding to the next
active node at level $j+1$. These replacement ciphertexts are created
with the keys used to create the original ciphertext pair. Note that
the distribution of this altered active node is identical to that of
the simulated node $n_j$ in the node chain described above. It is easy
to see that a garbled OBDD built with this definition has the same
distribution as 1) a real garbled OBDD when $i=0$ (i.e. for $H_0$), and
2) a simulated garbled OBDD when $i=B$ (i.e. for $H_B$).

We are now ready to show that the distribution of the simulated
garbled OBDD is indistinguishable from that of a real garbled OBDD; that
is, we will show that
$\{H_0(i_1,\ldots,i_n)\}=\{H_{B}(i_1,\ldots,i_n)\}$. Suppose to the
contrary that the distributions are distinguishable; that is, there
exists a poly-time distinguisher $\disting$ that 
\[
\begin{array}{l}
\vert \Pr[\disting(H_0(i_1,\ldots,i_n))=1] - \\
\Pr[\disting(H_{B}(i_1,\ldots,i_n))=1]>1/p\vert
\end{array}
\]
 for some polynomial
$p$. Then there exists a $j$ such that 
\[
\begin{array}{l}
\vert
\Pr[\disting(H_{j-1}(i_1,\ldots,i_n))=1] - \\
\Pr[\disting(H_{j}(i_1,\ldots,i_n))=1]>1/pB\vert
\end{array}
\]

Using $\disting$, we now build an adversary that breaks the semantic
security of the encryption scheme used to encrypt the garbled
nodes. Recall in a semantic security game, the adversary sends two
messages $m_0,m_1$ to the challenger and receives the encryption of
$m_b$ for $b=\binset$; the adversary's goal is to determine $b$. Let
$n^j$ be the $j$th node and we denote its two ciphertext terms as
$c^{j}_0$ and $c^{j}_1$. Note that node $n^j$ in the hybrid OBDD in
distribution $H^*_{j-1}$ is a real garbled node, whereas the same node
for distribution $H^*_{j}$ is a simulated garbled node; specifically,
$c^{j}_0$ and $c^{j}_1$ in distribution $H^*_{j-1}$ are encryptions of
different label-key messages, whereas they are encryptions of the same
label-key message in distribution $H^*_{j}$. We exploit this fact to
build the adversary $\adver$ that breaks semantic security of the
encryption scheme.

First, $\adver$ creates the hybrid garbled OBDD corresponding to the
distribution $H^*_{j-1}(i_1,\ldots,i_n)$. One of the two ciphertexts
$c^{j}_0$ and $c^{j}_1$ in node $n^j$ is an encryption of the label
and key for the active node $n^{j+1}$, whereas the other ciphertext is
an encryption of the label and key for an inactive node. Let $\ell_0$
be the label-key message encrypted in $c^j_0$ and $\ell_1$ be that
encrypted in $c^j_1$. Next, $\adver$ sends $\ell_0$ and $\ell_1$ to
the semantic security challenger and receives $c^*$, which is an
encryption of either $\ell_0$ or $\ell_1$. Without loss of generality,
let $\ell_0$ be the label-key message (contained in ciphertext
$c^{j}_0$) that leads to the next active node. $\adver$ replaces
$c^j_1$ with $c^*$ in node $n^j$ in the garbled OBDD that it built in
the first step, and then feeds the altered OBDD together with the other
required inputs to the hybrid distinguisher $\disting$. Note that
$c^*$ cannot be decrypted with the node and level keys for node
$n^j$. This fact, however, does not prevent the garbled OBDD from being
evaluated correctly because $c^*$ replaces $c^j_1$, which contains the
label and key to an inactive node, and would not be successfully
decrypted while evaluating the garbled OBDD on the inputs
$(i_1,\ldots,i_n)$.

$\disting$ eventually outputs a result stating that the input is of
distribution $H^*_{j-1}$ or $H^*_{j}$. If $\disting$ outputs that the
input is of distribution $H^*_{j-1}$, then $\adver$ outputs that $c^*$
is an encryption of $\ell_1$; otherwise $\adver$ outputs that $c^*$ is
an encryption of $\ell_0$. Notice that if $c^*$ is an encryption of
$\ell_0$, then both ciphertexts in node $n^j$ are encryptions of the
same label-message, and the input to $\disting$ has distribution
$H^*_{j}$. Similarly, if $c^*$ is an encryption of $\ell_1$, then the
input to $\disting$ has distribution $H^*_{j-1}$. Since $\disting$
distinguishes between $H^*_{j-1}$ and $H^*_{j}$ with non-negligible
probability, we see that $\adver$ wins the semantic security game with
non-negligible advantage. Since we assume that the encryption scheme is
semantically secure, this implication is a contradiction, and there is
no such distinguisher $\disting$ that distinguishes between
$H_0(i_1,\ldots,i_n)$ and $H_{B}(i_1,\ldots,i_n)$; that is, the
distribution of the simulated garbled OBDD is indistinguishable from
that of a real garbled OBDD. Therefore, Bob's simulated view is
indistinguishable, except with negligible probability, to the real
view, concluding the proof.
\end{proof}
