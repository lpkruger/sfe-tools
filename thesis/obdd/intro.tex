\section{Introduction}
\label{sec:intro}

% Importance of Privacy has spurred interest in privacy-preserving
% protocols.

The ease and transparency of information flow on the Internet has
heightened concerns of personal privacy~\cite{cra99,tur03}.  Various
Internet activities, such as Web surfing, email, and other
services leak sensitive information. As a result, there has been
interest in developing technologies~\cite{p3p02,gwb97,rks+97} and
protocols to address these concerns. In particular,
privacy-preserving protocols~\cite{FPRS04,FNP04,LP02,NPS99} that allow
multiple parties to perform joint computations without revealing their
private inputs have been the subject of much interest.
%Privacy-preserving protocols allow multiple parties with private
%inputs to perform joint computation without revealing their
%inputs.
Our focus in this paper is on two party privacy-preserving protocols.



% Secure function evaluation is a theoretical framework for privacy-preserving
% protocols. Two research thrusts have emerged.
% Making SFE more useable.
% Designing special purpose privacy-preserving algorithms.
One of the fundamental cryptographic primitives for designing
privacy-preserving protocols is {\it secure function evaluation
(SFE)}. A protocol for SFE enables two parties $A$ and $B$ with inputs
$x$ and $y$ respectively to jointly compute a function $f(x,y)$ while
preserving the privacy of the two parties (i.e., at the end of the
protocol, party $A$ only knows its input $x$ and the value of the
function $f(x,y)$, and similarly for $B$).  
Yao showed that for a polynomial-time computable function $f$, there
exists a SFE protocol that executes in polynomial
time~\cite{GMW87,Yao:86} (details about this protocol can be found in
Goldreich's book~\cite[Chapter 7]{Goldreich:vol2}). Yao's classic
solution for SFE has been used to design privacy-preserving protocols
for various applications~\cite{AMP04}.  The importance of Yao's
protocol spurred researchers to design a compiler that takes a
description of the function $f$ and emits code corresponding to Yao's
protocol for secure evaluation of $f$. Such compilers, for example
Fairplay~\cite{Fairplay}, enable wider applicability of SFE.
MacKenzie {\it et al.}~\cite{Reiter:CCS:2003} implemented a compiler for
generating secure two-party protocols for a restricted but important
class of functions, which is particularly suited for applications
where the secret key is protected using threshold
cryptography. For most applications, the classic protocol for SFE is
quite expensive, which has led researchers to develop more efficient
privacy-preserving protocols for specific
problems~\cite{FPRS04,FNP04,LP02,NPS99}.


% Mention importance of BDDs and their applications.
% Describe that it is graph-based data structure.
% Describe our basic contribution.
In the classic SFE protocol, the function $f$ is represented as
circuit comprised of gates. Fairplay uses this circuit
representation of $f$.  {\it Ordered Binary Decision Diagrams (OBDDS)}
are a graph-based representation of Boolean functions that have been
used in a variety of applications in computer-aided design, including
symbolic model checking (a technique for verifying designs),
verification of combinational logic, and verification of finite-state
concurrent systems~\cite{Bryant:BDD,Clarke:book}.  OBDDs can be
readily extended to represent functions with arbitrary domains and
ranges.  

Given an OBDD representation of the function to be jointly computed by
the two parties, Yao's protocol can be directly used by first converting the OBDD into a circuit.
Converting an OBDD to a circuit, however,
incurs a blow-up in the number of gates required. To empirically
measure this blowup, we implemented a compiler that takes an OBDD and
converts it into a circuit description that can be used in Fairplay.
On the average, this conversion from OBDD to circuit resulted in a
increase in size by a factor of $10$. Details of this experiment can
be found in Section~\ref{sec:experiments}.

In this paper, we present a
SFE algorithm that directly uses an OBDD representation of the
function $f$ that the two parties want to jointly compute. The
advantage of using an OBDD representation over the gate-representation
is that OBDDs are more succinct for certain widely used classes of
functions than the gate representation. For example, among other
functions, our results show the OBDD representation is more efficient
than the gate representation for 8-bit AND, 8-bit addition, and the
millionaire's and billionaire's problems~\cite{Yao:86}.  As a result,
our protocol has reduced bandwidth consumption over the classic Yao
protocol implemented in Fairplay. Because processor speeds have
increased at a more rapid pace than bandwidth availability over the
past years, network bandwidth is likely to be the bottleneck for a number of applications. In particular, our
protocols are especially useful for applications operating over
networks with limited bandwidth, such as wireless and sensor networks.
Furthermore, we have empirically confirmed this statement by
implementing our protocol and comparing it with Fairplay.

% List of contributions.
This paper makes the following contributions:
\begin{itemize}
\item We present a SFE protocol that uses the OBDD representation of
the function to be jointly computed by two parties. Our new protocol along with the
correctness proof is provided in Section~\ref{sec:sfe-obdd}.

\item 
Experimental results based upon a prototype implementation of our
protocol demonstrate that for certain functions, our
implementation results in a smaller encrypted circuit than
Fairplay. For example, for the classic millionaire's problem, our
implementation reduces the bandwidth by approximately $45$\% over Fairplay.  Our
implementation and experimental results are described in
Section~\ref{sec:experiments}.
\end{itemize}
In summary, this paper presents a new SFE protocol that uses the OBDD representation.
The OBDD representation is more efficient for several practical functions of 
interest. For other functions, the circuit description (and therefore
FairPlay) will be more efficient. This paper presents a generic alternative to 
Boolean circuits that can be used when appropriate. 


