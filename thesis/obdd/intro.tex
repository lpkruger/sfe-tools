\section{Contribution}
\label{sec:obdd-intro}

% Importance of Privacy has spurred interest in privacy-preserving
% protocols.

% Mention importance of BDDs and their applications.
% Describe that it is graph-based data structure.
% Describe our basic contribution.

{\it Ordered Binary Decision Diagrams (OBDDS)}, introduced and
described in section 3.1, are a graph-based representation of Boolean
functions.  In this chapter, we present an
SFE algorithm that directly uses an OBDD representation of the
function $f$ that the two parties want to jointly compute. 

% List of contributions.
This dissertation presents the following contributions:
\begin{itemize}
\item We present a SFE protocol that uses the OBDD representation of
the function to be jointly computed by two parties. The new protocol along with the
correctness proof is provided in Section~\ref{sec:sfe-obdd}.

\item 
Experimental results based upon a prototype implementation of our
protocol demonstrate that for certain functions, our
implementation results in a smaller encrypted circuit than
the equivalent Yao circuit. For example, for the classic millionaire's problem, our
implementation reduces the bandwidth by approximately $45$\% over the Yao
protocol.  The
implementation and experimental results are described in
Section~\ref{sec:experiments}.
\end{itemize}

The
advantage of using an OBDD representation over the gate-representation
is that OBDDs are more succinct for certain widely used classes of
functions than the gate representation. For example, among other
functions, our results show the OBDD representation is more efficient
than the gate representation for 8-bit AND, 8-bit addition, and the
millionaire's and billionaire's problems~\cite{Yao:86}.  As a result,
our protocol has reduced bandwidth consumption over the classic Yao
protocol.  Because processor speeds have
increased at a more rapid pace than bandwidth availability over the
past years, network bandwidth is likely to be the bottleneck for a number of applications. In particular, our
protocols are especially useful for applications operating over
networks with limited bandwidth, such as wireless and sensor networks.
In section~\ref{sec:experiments}, we empirically confirm this statement by
implementing our protocol and comparing it with the Yao protocol.

In summary, this dissertation presents a new SFE protocol that uses the OBDD representation.
The OBDD representation is more efficient for several practical functions of 
interest. For other functions, the circuit description (and therefore
FairPlay) will be more efficient. This protocol presents a generic alternative to 
Boolean circuits that can be used when appropriate. 


