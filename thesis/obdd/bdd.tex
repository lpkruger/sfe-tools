\section{Overview}
\label{sec:OBDDs}

{\it Ordered binary decision diagrams (OBDDs)}, introduced in section 3.1, are a canonical 
representation for Boolean formulas~\cite{Bryant:BDD}. They are often
substantially more compact than traditional normal forms, such as
conjunctive normal form (CNF) and disjunctive normal form (DNF), and
they can be manipulated efficiently. Therefore, they are widely used for a
variety of applications in computer-aided design, including symbolic
model checking, verification of combinational logic, and verification of
finite-state concurrent systems~\cite{Clarke:book}.  A detailed
discussion of OBDDs can be obtained in Bryant's seminal
article~\cite{Bryant:BDD}.


Given a Boolean function $f(x_1,x_2,\cdots,x_n)$ of $n$ variables
$x_1, \cdots, x_n$ and a total ordering on the $n$ variables, the OBDD
for $f$, denoted by $OBDD(f)$, is a rooted, directed acyclic graph
(DAG) with two types of vertices: {\it terminal} and {\it nonterminal}
vertices. $OBDD(f)$ also has the following components:
\begin{itemize}
\item Each vertex
$v$ has a level, denoted by ${\it level}(v)$, between $0$ and $n$. There is a 
distinguished vertex called  {\it root} whose level is $0$. 

\item Each nonterminal vertex $v$ is labeled by a variable ${\it var}(v) \in \{ x_1,\cdots,x_n \}$ and 
has two successors, ${\it low (v)}$ and ${\it high (v)}$. Each
terminal vertex is labeled with either $0$ or $1$. There are only two terminal vertices 
in an OBDD. Moreover, the labeling of vertices respects the total ordering $<$ on the
variables, i.e., if $u$ has a nonterminal successor $v$, then ${\it var}(u) < {\it var}(v)$. 
\end{itemize}
Given an assignment ${\cal A} \; = \; \langle x_1 \leftarrow b_1,
\cdots, x_n \leftarrow b_n \rangle$ to the variables $x_1,\cdots,x_n$
the value of the Boolean function $f(b_1,\cdots,b_n)$ can be obtained by
starting at the root and following the path where the edges on the
path are labeled with $b_1, \cdots, b_n$. OBDDs can also be used to
represent functions with finite range and domain. Let $g$ be a
function of $n$ Boolean variables with output that can be encoded by
$k$ Boolean variables. The function $g$ can be represented as an array
of $k$ OBDDs where the $i$-th OBDD represents the Boolean function
corresponding to the $i$-th output bit of $g$.  For the rest of the
paper we will assume that the function $f$ is a Boolean function, but
our protocols can be easily extended for the case of functions with a
finite range. We will illustrate OBDDs with an example.
\begin{example}
\label{example:bdd}
\rm
Figure~\ref{fig:OBDD} shows the OBDD for the  function
$f(x_1,x_2,x_3, x_4) \; = \; (x_1 \; = \; x_2) \wedge (x_3 \; = \;
x_4)$ of four variables $x_1,x_2,x_3,x_4$ with the total ordering $x_1
< x_2 < x_3 < x_4$.\footnote{OBDDs are sensitive to variable
ordering, e.g., with the ordering $x_1 < x_3 < x_2 < x_4$ the OBDD for
$(x_1 \; = \; x_2) \wedge (x_3 \; = \; x_4)$ has $11$ nodes.}  Notice that the ordering of
the labels on the vertices on any path from the root to the terminals
of the OBDD corresponds to the total ordering of the Boolean
variables. Consider the assignment $\langle x_1 \leftarrow 1, x_2
\leftarrow 1, x_3 \leftarrow 0, x_4 \leftarrow 0 \rangle$.  In the
OBDD shown in Figure~\ref{fig:OBDD}, if we start at the root and
follow the edges corresponding to the assignment, we end up at the
terminal vertex labeled with $1$. Therefore, the value of $f(1,1,0,0)$
is $1$.
\end{example}

\begin{figure}
\begin{minipage}{3in}
\centering
\fbox{\epsfysize=3.2in \epsfbox{obdd/figures/bdd-fig1.pdf}}
\caption{OBDD for the function $f(x_1,x_2,x_3, x_4) \; = \; (x_1 \; = \; x_2) \wedge (x_3 \; = \;
x_4)$.}
\label{fig:OBDD}
\end{minipage}
\hfill
\begin{minipage}{3in}
\centering
\fbox{\epsfysize=2.5in \epsfbox{obdd/figures/bdd-fig2.pdf}}
\caption{OBDD for the restriction $\;\;\;\;$ $f\mid_{x_1 \leftarrow 1, x_3 \leftarrow 0}$ where
 $f(x_1,x_2,x_3, x_4) \; = \; (x_1 \; = \; x_2) \wedge (x_3 \; = \; x_4)$.}
\label{fig:OBDD-reduced}
\end{minipage}
\end{figure}


One of the advantages of OBDDs is that they can be manipulated
efficiently, i.e., given OBDDs for $f$ and $g$, OBDDs for $f \wedge
g$, $f \vee g$, and $\neg f$ can be computed efficiently. We now
describe an operation called {\it restriction}, which is used in our
protocol.  Given a $n$ variable Boolean function
$f(x_1,x_2,\cdots,x_n)$ and a Boolean value $b$, $f
\mid_{x_i \leftarrow b}$ is a Boolean function of $n-1$ variables
$x_1, \cdots, x_{i-1},x_{i+1},\cdots,x_n$ defined as follows:\\
$f \mid_{x_i \leftarrow b} (x_1, \cdots, x_{i-1},x_{i+1},\cdots,x_n)$ is
equal to $f(x_1, \cdots, x_{i-1},b,x_{i+1},\cdots,x_n)$.
Essentially $f \mid_{x_i \leftarrow b}$ is the function obtained by substituting
the value $b$ for the variable $x_i$ in the function $f$. 
Given the OBDD for $f$, the $OBDD$ for $f \mid_{x_i \leftarrow b}$ can
be efficiently computed~\cite[Section 4]{Bryant:BDD}.  The restriction
operation can be extended to multiple variables in a
straightforward manner, e.g., $f \mid_{x_i \leftarrow b, x_j \leftarrow b'}$
can be computed as $(f \mid_{x_i \leftarrow b}) \mid_{x_j \leftarrow b'}$. We
explain the algorithm using our example; the reader is referred to~\cite{Bryant:BDD}
for details. 
Consider the function $f(x_1,x_2,x_3,x_4)$
described in example~\ref{example:bdd}. The OBDD corresponding to $f
\mid_{x_1 \leftarrow 1, x_3 \leftarrow 0}$ is shown in
Figure~\ref{fig:OBDD-reduced}.  
Since $x_1 \leftarrow 1$, the root of OBDD $(f\mid_{x_1 \leftarrow 1, x_3 \leftarrow 0})$
is the left vertex labeled with $x_2$. Consider the two vertices $v_1$
and $v_2$ labeled with $x_2$. If $v_1$ has an edge that points to the vertex labeled with $x_3$, then
that edge is changed to point to the right vertex labeled with $x_4$ (because this is the vertex
reached if $x_3$ is equal to $1$). 
Notice that in the reduced OBDD shown in Figure~\ref{fig:OBDD-reduced}
the vertices that are labeled with $x_1$ and $x_3$ have been eliminated. 

