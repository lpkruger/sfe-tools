
\section{Comparison with the edit distance protocol of~\cite{atallah}}
\label{appendix-atallah}

In~\cite{atallah}, Atallah \emph{et al.} presented a privacy-preserving
edit distance protocol, which is superficially similar to our Protocol 2
in that the intermediate values $D(i,j)$ are additively shared between
Alice and Bob.  The protocol of~\cite{atallah} relies on different
cryptographic techniques, including special-purpose solutions to the so
called ``millionaires' problem'' (a two-party protocol, in which the
parties determine whose input is bigger without revealing the actual
input values) and additively homomorphic encryption.

In this section, we present a detailed comparison of the online
computational cost of our protocol vs.\ that of~\cite{atallah}.  Let $q
= \lceil \log w \rceil$ be the length of each alphabet symbol, and
let $s=\log(n+m)$ be the length of random shares used to mask $D(i,j)$
in our protocol.  In the protocol of~\cite{atallah}, masking is done by
adding random values under encryption, in a group of unknown order which
is much larger than $2^{n+m}$.  Therefore, this addition \emph{cannot}
be modular, and must be done over the integers.  To achieve standard
cryptographic security, the length of random shares in bits must be at
least $s'=\log(n+m)+80=s+80$.

Below, we compare the cost for a single iteration, since the number of
iterations is equal to $n \times m$ in both protocols.

\vspace{1ex}
\noindent
\textbf{Online computational cost of~\cite{atallah}.}
Each iteration uses the minimum or maximum finding protocol three times:
twice in step 1 on $q$-bit values, and once in step 5 on $s'$-bit
values~\cite[section 4.1]{atallah}.  Each minimum/maximum finding
protocol requires two instances of the millionaires' sub-protocol, and
six re-randomizations of Paillier ciphertexts.  The latter is done by
exponentiation modulo $N^2$, where $N^2$ is the modulus of an instance
of Paillier encryption scheme.  $N$ itself is an RSA modulus and must
be at least 1024 bits; therefore, $N^2$ is at least 2048 bits.

The implementations of the millionaires' protocol suggested
in~\cite{atallah} are relatively inefficient.  For fair comparison,
we will assume that the construction of~\cite{atallah} is instantiated
with a state-of-the-art sub-protocol for the millionaires' problem, \eg,
the Lin-Tzeng protocol~\cite{lintzeng-acns05}.  This protocol requires
$(1540 s' - 6)$ online modular multiplications per instance if $s'$-bit
values are being compared ($1540 q - 6$ if $q$-bit values are being
compared), assuming the standard size of 512 bits for the prime moduli
in ElGamal encryption.

Assuming that the permutations required by~\cite{atallah} are free,
the online cost of each iteration is thus equivalent to
$2 \times (2 \times (1540 q - 6) + 6 \times 2048) +
          (2 \times (1540 s' - 6) + 6 \times 2048)$ =
$2 \times (3080 q - 12 + 12288) +
          (3080 s' - 12 + 12288)$ =
$3080s' + 6160q + 36828$ =
$3080s + 6160q + 283228$ modular multiplications.

\vspace{1ex}
\noindent
\textbf{Online computational cost of our Protocol 2.}

Each iteration of our Protocol 2 involves evaluation of several ``garbled
circuits.''  Each $\CE$ circuit has $2q$ gates of arity 2, and each
$\CM$ circuit has $10s$ gates of arity 2, and $5s-6$ gates of arity 3.
In each iteration, a single instance of $\CE$ and a single instance of
$\CM$ must be evaluated (in our presentation, evaluation of circuits $\CE$
and $\CM$ is split between two phases, but there is a 1:1 correspondence
between the iterations of each phase).

All garbled circuits can be pre-computed in advance, because the
representation of the circuit in Yao's protocol is independent of the
actual input values.  Each row of the truth table of each gate becomes
a double-encrypted symmetric ciphertext (see section~\ref{yao}), for
a total of $4 \times 2 q + (4 \times 10s +  8 \times (5s - 6))$ = $8q
+ 80s - 48$ ciphertexts.  Decrypting each double-encrypted ciphertext
requires two online symmetric decryptions, but, on average, the evaluator
of a garbled gate will only need to try decrypting half the ciphertexts
before decryption succeeds and he obtains the wire key representing the
bit value of the gate's output wire.

Transferring the wire-key representation of Bob's $q$-bit input into $\CE$
requires $q$ instances of $OT_1^2$.  The online cost of each instance
is 2 modular exponentiations and 1 modular exponentiations.  Therefore,
assuming 512-bit moduli, the total online cost of obliviously transferring
the inputs to $\CE$ is equivalent to $1025q$ modular multiplications.

In the same iteration, a single instance of $\CM$ must be evaluated.
Bob has three $s$-bit inputs (after evaluating $\CE$, he already has
the representation for his fourth input).  Obliviously transferring
the wire-key representation of these inputs requires $3s$ instances of
$OT_1^2$, for a total cost of $3075 s$ modular multiplications.

Therefore, the total online cost of each iteration of our Protocol
2 is $(3075 s + 1025 q)$ modular multiplications and $8q + 80s -
48$ symmetric decryptions vs.\ $(3080 s + 6160 q + 283228)$ modular
multiplications in each iteration of~\cite{atallah}.  Since symmetric
decryption is much cheaper than modular multiplication, we conclude
that our Protocol 2 offers significantly better efficiency than the
protocol of~\cite{atallah}.  In general, the protocol of~\cite{atallah}
requires \emph{at least} 300,000 modular multiplications per iteration,
rendering it unrealistically expensive for practical applications.

