\section{Cryptographic Toolkit}
\label{crypto}

We will employ several standard cryptographic techniques.

\vspace{1ex}
\noindent
\textbf{Oblivious transfer.}
\emph{Oblivious transfer} was originally proposed by Rabin~\cite{R81}.
Informally, a $1$-out-of-$n$ oblivious transfer (denoted as $OT_1^n$)
is a protocol between two parties, the chooser and the sender.
The sender's inputs into the protocol are $n$ values $v_1,\ldots,v_n$.
The chooser's input is an index $i$ such that $1 \leq i \leq n$.
As a result of the protocol, the chooser receives $v_i$, but does not
learn anything about the rest of the sender's values.  The sender learns
nothing.  Our protocols do not depend on a particular implementation of
oblivious transfer; therefore, we simply assume that we have access to a
cryptographic primitive implementing $OT_1^2$.  In our implementations,
we rely on Fairplay~\cite{Fairplay} and the Naor-Pinkas oblivious transfer
construction~\cite{Naor-Pinkas:2001}.

\vspace{1ex}
\noindent
\textbf{Oblivious circuit evaluation.}
\label{yao}
We also employ two standard methods for secure circuit evaluation: Yao's
``garbled circuits'' method and secure computation with shares.  Consider
any (arithmetic or Boolean) circuit $C$, and two parties, Alice and Bob,
who wish to evaluate $C$ on their respective inputs $x$ and $y$.  

Yao's ``garbled circuits'' method was originally proposed in~\cite{Yao86}
(a complete description and security proofs can be found in~\cite{LP04}).
Informally, Alice securely transforms the circuit so that Bob can
evaluate it without learning her inputs or the values on any internal
circuit wire except the output wires.

Alice does this by generating two random keys for each circuit wire,
one representing $0$ on that wire, the other representing $1$.  The keys
representing Alice's own inputs into the circuit she simply sends to
Bob.  The keys representing Bob's inputs are transferred to Bob via the
$OT_1^2$ protocol.  For each of Bob's input wires, Bob acts as the chooser
using his input bit on that wire as his input into $OT_1^2$, and Alice
acts as the sender with the two wire keys for that wire as her inputs
into $OT_1^2$.  If Bob has a $q$-bit input into the circuit, then $q$
instances of $OT_1^2$ are needed to transfer the wire keys representing
his input, since each input bit is represented by a separate key.

Alice produces the ``garbled'' truth table for each circuit gate in
such a way that Bob, if he knows the wire keys representing the values
on the gate input wires, can decrypt exactly one row of the garbled
truth table and obtain the key representing the value of the output wire.
For example, consider an AND gate whose input wires are $a$ and $b$,
and whose output wire is $c$.  Let $k^0_a,k^1_a,k^0_b,k^1_b,k^0_c,k^1_c$
be the random wire keys representing the bit values on these wires.
The garbled truth table for the gate is a random permutation of
the following four ciphertexts:
$E_{k^1_a}(E_{k^0_b}(k^0_c))$,
$E_{k^1_a}(E_{k^1_b}(k^1_c))$.
$E_{k^0_a}(E_{k^1_b}(k^0_c))$,
$E_{k^0_a}(E_{k^0_b}(k^0_c))$.
Yao's protocol maintains the invariant that for every circuit wire,
Bob learns \emph{exactly one} wire key.

Because wire keys are random and the mapping from wire keys to values
is not known to Bob (except for the wire keys corresponding to his own
inputs), this does not leak any information about actual wire values.
The circuit can thus be evaluated ``obliviously.''  For example, given
the above table and the input wire keys $k^0_a$ and $k^1_b$ representing,
respectively, $0$ on input wire $a$, and $1$ on input wire $b$, Bob
can decrypt exactly one row of the table, and learn random key $k^0_c$
representing $0$ (\ie, the correct result of evaluating the gate) on
the output wire $c$.

Observe that until Alice reveals the mapping, Bob does \emph{not}
know which bits are represented by the wire keys he holds.  For the
standard garbled circuit evaluation, Alice reveals the mapping only
for the wires that represent the output of the entire circuit, but
not for the intermediate wires.

Several of our protocols rely on the representation of bit values on
circuit wires by random keys.   These protocols use Yao's construction
not as a ``black box'' implementation of secure circuit evaluation,
but exploit its internal structure in a fundamental way.

The second standard method is \emph{secure computation with shares}
(SCWS)~\cite[Chapter 7]{Goldreich:vol2}.  This protocol maintains
the invariant that, for every circuit wire $w$, Alice learns a random
value $s$ and Bob learns $b_w - s$, where $b_w$ is the bit value of
the wire.  Therefore, Alice's and Bob's shares add up to $b_w$, but
because the shares are random, neither party knows the actual wire value.
For each output wire of the circuit, Alice and Bob combine their shares to
reconstruct the circuit output.  

% Either Yao's ``garbled circuits'' method,
% or SCWS can be used to securely and privately evaluate any circuit $C$.

