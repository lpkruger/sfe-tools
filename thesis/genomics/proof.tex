\section{Security Proofs}

\newcommand{\view}{\mathsf{view}}
\newcommand{\sot}{S^\mathsf{ot}}
\newcommand{\mot}{S^\mathsf{min}}

Our protocols are secure in the so called the \emph{semi-honest} model
of secure computation, \ie, under the assumption that both participants
faithfully follow the protocol specification.  To achieve security in
the \emph{malicious} model, where participants may deviate arbitrarily
from protocol specification, participants would need to commit to their
respective inputs prior to protocol start and then prove in zero knowledge
that they follow the protocol specification.  

Since we use Yao's ``garbled circuits'' method as the underlying
primitive, security in the malicious model, if needed, can be achieved as
at constant cost~\cite{JS07}.  For practical usage scenarios, however, it
is not clear whether security in the malicious model offers significant
advantages over security in the semi-honest model.  For example, there
is no external validation of the parties' inputs.  Even if the protocol
forces each party to run the protocol on previously committed inputs,
this does not guarantee that the inputs are not maliciously chosen in
the first place.  In other words, a malicious party may simply commit
to a ``bad'' input (deliberately chosen so that the result of the edit
distance computation reveals some information about the other party's
input) and pass all proofs.

In general, we expect that our protocols will be used for tasks such as
collaborative analysis of genome sequence in joint medical studies, where
it is reasonable to assume that participants provide actual sequences
as inputs into the protocol, and are not deliberately supplying fake
sequences in an attempt to learn something about the other participant's
data.

Security of Protocol 1 follows directly from (i) security of subprotocols
performed using standard methods for secure multi-party computation,
and (ii) composition theorem for the semi-honest model~\cite[Theorem
7.3.3]{Goldreich:vol2}.  Proofs are standard and omitted for brevity.

Security of Protocol 2 is proved via a standard simulation in the
semi-honest model.  For each protocol participant, we demonstrate the
existence of an efficient simulator algorithm which, with access to
this participant's input and output, produces a simulation which is
computationally indistinguishable from this participant's ``view'' of
the protocol (informally, a ``view'' is a record of sent and received
messages).

Let $\view_A(\alpha)$ (respectively, $\view_B(\beta)$) be Alice's
(respectively, Bob's) view of the protocol when executed on input string
$\alpha$ (respectively, $\beta$).  Each party's view consists of its
respective input as well as all messages received by this party in the
course of the protocol.  The output of the protocol is the edit distance
$\delta(\alpha,\beta)$.  Because edit distance is a deterministic function
of the parties' inputs, to prove security of the protocol it is sufficient
to construct simulators $S_A$ and $S_B$ such that
\[
\begin{array}{rcl}
\{S_A(\alpha,\delta(\alpha,\beta))\} 
& \stackrel{c}{\equiv} &
\{\view_A(\alpha)\} \\
\{S_B(\beta,\delta(\alpha,\beta))\} 
& \stackrel{c}{\equiv} &
\{\view_B(\beta)\}
\end{array}
\]
Here $\stackrel{c}{\equiv}$ stands for computational
indistinguishability~\cite{GoldreichBookVol1}.

As the building blocks for our simulator, we will use the simulators for
Yao's ``garbled circuits'' protocols for evaluating the $\CE'$ and $\CM$
circuits.  The difference between $\CE'$ and $\CE$ is that, unlike $\CE$,
which outputs the wire key representing the result of equality testing
(as opposed to the actual result), $\CE'$ outputs a single bit $\sigma$:
$0$ if the values are equal, $1$ if they are not (\ie, $\CE'$ is the
standard equality testing circuit).

% Because the oblivious transfer protocol is assumed to be secure, there
% exist simulators $\sot_A,\sot_B$ such that, for $i\in\{0,1\}$:
% \[
% \begin{array}{rcl}
% \{\sot_A(i,x_i)\} 
% & \stackrel{c}{\equiv} &
% \{\view^{\mathsf{ot}}_A(i)\} \\
% \{\sot_B(x_0,x_1,\perp)\} 
% & \stackrel{c}{\equiv} &
% \{\view^{\mathsf{ot}}_B(x_0,x_1)\}
% \end{array}
% \]
% where $x_0,x_1$ are Bob's inputs into the $OT_1^2$ protocol,
% $i$ is Alice's choice ($0$ or $1$), $\perp$ is the ``empty'' output
% (it denotes that Bob does not receive any output from the protocol), and
% $\view^{\mathsf{ot}}_A$ and $\view^{\mathsf{ot}}_B$ are, respectively,
% Alice's and Bob's views of the $OT_1^2$ protocol.

Security of the protocol for computing the $\CM$ circuit implies that
there exist simulators $\mot_A,\mot_B$ such that:
\[
\begin{array}{rcl}
\{\mot_A(x_1,x_2,x_3,r)\} 
& \stackrel{c}{\equiv} &
\{\view^{\mathsf{min}}_A(x_1,x_2,x_3,r)\} \\
\{\mot_B(y_1,y_2,y_3,t,z)\} 
& \stackrel{c}{\equiv} &
\{\view^{\mathsf{min}}_B(y_1,y_2,y_3,t)\} \\
\multicolumn{3}{l}{
\mbox{where } z=\min(x_1 \oplus y_1 + 1,x_2 \oplus y_2 + 1,x_3 \oplus y_3+t) \oplus r
}
\end{array}
\]

The simulators for the parties' respective views of $\CE'$ are similar.


\paragraph{Simulating Alice's view.}
Simulation of Alice's view in Phase 0 is trivial.

In Phase 1, Alice participates in multiple instances of secure
evaluation of circuits $\CE$.  By security of Yao's ``garbled circuits''
protocol~\cite{LP04}, there exist simulators for Alice's views of every
instance of $\CE'$.  Since Alice's view of $\CE$ is exactly the same as
her view of $\CE'$ (the only difference between the two circuits is their
respective outputs for Bob), our simulator simply invokes the simulator
for $\CE'$ to simulate Alice's view of each instance.

Phase 2 consists of $n \times m$ iterations, one for each value of the
$(i,j)$ pair.  Therefore, Alice's $\view_A$ of Protocol 2 is a composition
of Alice's views of individual iterations $\view_A^{(i,j)}$.

For all $(i,j)$ where either $i \neq n$, or $j \neq m$, $\view_A^{(i,j)}$
is simply her view of the secure evaluation protocol for the circuit
$\CM$.  To simulate Alice's view of the $\CM$ evaluation on the $(i,j)$
instance, the simulator simply invokes the sub-simulator $\mot_A$ for
this protocol.  Note that Alice receives no output from $\CM$.

Finally, $\view_A^{(n,m)}$ contains an additional message $m_B$ from
Bob at the very end of the protocol, which in the real execution
enables Alice to reconstruct the output of the entire protocol,
\ie, $\delta(\alpha,\beta)$.  Because the simulator has access to
$\delta(\alpha,\beta)$, it simulates $m_B$ as $\delta(\alpha,\beta)-r_A$,
where $r_A$ is Alice's fourth input into the $\CM$ circuit in the
$(n,m)$ iteration.  Observe that in both the simulation and the real
execution, the sum of $r$ and $m_B$ is equal to $\delta(\alpha,\beta)$.
This completes the simulation of Alice's view.


\paragraph{Simulating Bob's view.}
Simulating Bob's view is a little more difficult.  The simulator
maintains an internal table $M: \{0,1\}^r \rightarrow \{0,1\}$.
In Phase 1, for each instance of circuit $\CE(i,j)$, our simulator
invokes the sub-simulator for Bob's view of $\CE'(i,j)$ on Bob's input
$\beta[j]$.  The output of the sub-simulator is the bit $\sigma(i,j)$,
which represents whether $\alpha[i]=\beta[j]$ or not.  

The simulator generates a random $r$-bit value $k(i,j)$, stores the
mapping $M(k(i,j))=\sigma(i,j)$ in its internal table $M$, and sends
$k(i,j)$ to Bob.  Now, in the real evaluation of $\CE$, Bob receives an
$r$-bit wire key.  Because wire keys are generated uniformly at random,
the simulated value $k(i,j)$ has exactly the same distribution as the
real wire key, and Bob cannot distinguish between the key from the real
execution and the simulation.

In Phase 2, for each instance of circuit $\CM(i,j)$, the simulator runs
on Bob's input $k(i,j)$, which is equal to the same random value that the
simulator returned to Bob when simulating $\CE(i,j)$.  Bob's inputs are
$D_B(i-1,j)$,$D_B(i,j-1)$,$D_B(i-1,j-1)$, and $k(i,j)$.  Our simulator
invokes the sub-simulator $\mot_B(D_B(i-1,j),D_B(i,j-1),D_B(i-1,j-1),
M(k(i,j)))$.  Observe that the fourth argument is the result of looking
up $\sigma(i,j)$ corresponding to $k(i,j)$ in the simulator's internal
table $M$.  Recall that $\sigma(i,j)$ is the result of the comparison
between $\alpha[i]$ and $\beta[j]$.  The output of $\mot_B$ is returned
to Bob.

Finally, $\view_B^{(n,m)}$ contains an additional message $m_A$ from
Alice at the very end of the protocol, which in the real execution
enables Bob to reconstruct the output of the entire protocol,
\ie, $\delta(\alpha,\beta)$.  Because the simulator has access to
$\delta(\alpha,\beta)$, it simulates $m_A$ as $\delta(\alpha,\beta)-r_B$,
where $r_B$ is the output of the $\mot_B$ sub-simulator in the $(n,m)$
iteration.

Indistinguishability of Bob's real and simulated views follows from the
existence of simulators for $\CE'$ and $\CM$, and the fact that Bob
cannot tell the difference between a wire key generated by Alice (in
the real protocol) and a ``fake'' value of the same length generated
randomly by the simulator (in the simulated protocol), since both are
random values are drawn from the same distribution.

% This sub-simulator requires the actual
% minimum value as one of its inputs.  The simulator substitutes a random
% value $r''$ for the actual minimum.  As in any protocol for securely
% evaluating function $f(x,y)$ where Alice and Bob hold random shares of
% $x$ and $y$~\cite{Goldreich:vol2}, Alice's share of the result is random
% and independent of $f(x,y)$.  Therefore, the substitution is undetectable.

