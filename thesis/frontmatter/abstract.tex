Secure function evaluation (SFE) is a cryptographic technique for evaluating
functions among multiple parties while preserving the privacy of parties'
inputs.  Traditional SFE techniques can impose high performance penalties 
compared to non-privacy preserving protocols, and SFE has largely been a
theoretical curiosity without much practical application.  In this thesis, 
we demonstrate techniques for improving the performance of SFE.  We present
a new protocol for SFE using Ordered Binary Decision Diagrams, and also we
show how to design optimized protocols for dynamic programming problems.
Our techniques are demonstrated on k-means clustering and the Smith-Waterman
gene sequence alignment algorithms.  All of our protocols are implemented and
evaluated for real-world performance, and are available for download.
